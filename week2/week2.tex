\documentclass{tufte-book}

\usepackage{amsmath, amsthm}
\usepackage{graphicx}
\setkeys{Gin}{width=\linewidth,totalheight=\textheight,keepaspectratio}
\graphicspath{{graphics/}}

\title{Real Analysis\\Second Week }
\author{Joe Seidel}
\date{\today}

\usepackage{booktabs}
\usepackage{units}
\usepackage{fancyvrb}
\fvset{fontsize=\normalsize}
\usepackage{multicol}
\usepackage{lipsum}
\usepackage{pdfpages}
\usepackage{tikz}
\usepackage{wasysym}

\newcommand{\doccmd}[1]{\texttt{\textbackslash#1}}% command name -- adds backslash automatically
\newcommand{\docopt}[1]{\ensuremath{\langle}\textrm{\textit{#1}}\ensuremath{\rangle}}% optional command argument
\newcommand{\docarg}[1]{\textrm{\textit{#1}}}% (required) command argument
\newenvironment{docspec}{\begin{quote}\noindent}{\end{quote}}% command specification environment
\newcommand{\docenv}[1]{\textsf{#1}}% environment name
\newcommand{\docpkg}[1]{\texttt{#1}}% package name
\newcommand{\doccls}[1]{\texttt{#1}}% document class name
\newcommand{\docclsopt}[1]{\texttt{#1}}% document class option name


\newtheoremstyle{mytheoremstyle} % name
	{\topsep}		% Space above
	{\topsep}		% Space below
	{\itshape}		% Body font
	{}			% Indent amount
	{\bfseries}	% Theorem head font
	{\textnormal{:}}	% Punctuation after theorem head
	{.5em}		% Space after theorem head
	{}			%Theorem headspec 
\theoremstyle{mytheoremstyle}
\newtheorem*{thm}{Thm.}

\newtheoremstyle{mylemstyle} % name
	{\topsep}		% Space above
	{\topsep}		% Space below
	{\itshape}		% Body font
	{}			% Indent amount
	{\bfseries}	% Theorem head font
	{\textnormal{:}}	% Punctuation after theorem head
	{.5em}		% Space after theorem head
	{}			%Theorem headspec 
\theoremstyle{mylemstyle}
\newtheorem*{lem}{Lem.}


\newtheoremstyle{mydefstyle} % name
	{\topsep}		% Space above
	{\topsep}		% Space below
	{\normalfont}	% Body font
	{}			% Indent amount
	{\bfseries}	% Theorem head font
	{\textnormal{:}}	% Punctuation after theorem head
	{.5em}		% Space after theorem head
	{}			%Theorem headspec 
\theoremstyle{mydefstyle}
\newtheorem*{mydef}{Def.}
\newtheorem*{ex}{E.g.}

\begin{document}

\maketitle
\pagenumbering{gobble}
\newpage
\pagenumbering{arabic}

\subsection{Exercise 1.6.35}

\begin{enumerate}
\item Show the $\emptyset$ and $\mathbb{R}$ are the only subsets of $\mathbb{R}$ that are both open and closed.

\begin{proof}
Let S be a non-empty open and closed set of $\mathbb{R}$.  Fix $x_0 \in S$ and $S \neq \mathbb{R}$.  Then, $\exists y \in \mathbb{R} \setminus S$.  Without loss of generality we may assume $y > x_0$.

Therefore we can form a set
\[I = \{ x \in \mathbb{R} | x > x_0, x \notin S\}\]

By construction\marginnote{greatest lower bound} this set is bounded below by $x_0$ and not empty because $y > x_0$, therefore $y \in I$.  Therefore we can let $i = \inf I$

Suppose $i \in S$.  Since $S$ is open it contains the open interveral $(i-\epsilon, i+\epsilon)$ for $\epsilon > 0$.  However, this interval contradicts that $i = \inf I$ because it implies a sequence $i_n > i$, $|i - i_n| < \frac{1}{n}$, where $i_n \in I$, which means $i_n \notin S$.  Which is not possible because $[i, i+e) \subset S$

Now suppose $i \notin S$.  Since $S$ is closed, $S^c$ which means that it contains an open interval $(i-\epsilon, i+\epsilon)$, but this contradicts the definition $i = \inf I$ because then we can find $i - \frac{e}{2}$ that is in $S^c$.

Therefore $S = \mathbb{R}$.
\end{proof}

\item Show that every non-empty open set in $\mathbb{R}$ can be written as a countable union of pairwise disjoint open intervals.

\begin{proof}

Let $S$ be an open interval in $\mathbb{R}$.  We can choose and arbitrary $x \in \mathbb{Q}$ inside $S$ and choose $r$ and $r'$ both greater than $0$, such that $(x-r, x+r')$ covers the $S$.  Since $\mathbb{Q}$ is countable, we can effectively cover any open interval in $R$ by a countable union of pairwise disjoint open intervals.

\end{proof}

\item Show that an arbitrary union of open sets in $\mathbb{R}$ is open in $\mathbb{R}$.

\begin{proof}
Suppose $\{Aß_i \subset \mathbb{R} | i \in I\}$ is an arbitrary collection of open sets.

If $x \in \cup A_i$ then $x \in A_i$ for some $i \in I$.  Since $A_i$ is open,  there $(x - \epsilon, x+ \epsilon) \subset A_i$ for $\epsilon > 0$.  Therefore,
\[ (x-\epsilon, x+\epsilon) \subset \bigcup\limits_{i \in I}A_i \]

 since $A_i \subset \mathbb{R}$ we arrive at our conclusion: The arbitrary union of open sets, $\bigcup\limits_{i \in I}A_i$ is open in $\mathbb{R}$. 
\end{proof}

\item Show that a finite intersection of open sets in $\mathbb{R}$ is open in $\mathbb{R}$
\begin{proof}
Suppose $\{A_i \subset \mathbb{R} | i = 1,2,...n\}$ is a finite collection of open sets.  If $ x \in \bigcap\limits_{i=1}^{n} A_i $ then $x \in A_i$ for every $1 \leq i \leq n$ Since $A_i$ is open, there are $\epsilon_i > 0$  such that $(x - \epsilon_i , x+ \epsilon_i) \subset A_i$

Let $\epsilon = min(\epsilon_1,..., \epsilon_n) > 0$.  This shows $(x - \epsilon, x+ \epsilon) \subset  \bigcap\limits_{i=1}^{n} A_i $

\end{proof}

\item Show, by example, that an infinite intersection of open sets is not necessarily open.
The open interval $(- \frac{1}{n},\frac{1}{n})$ is open $\forall n \in \mathbb{N}$  However, it's intersection
\[ \bigcap\limits_{n=1}^{\infty}(- \frac{1}{n},\frac{1}{n}) = {0} = [0..0] \]
which is closed. 

\item Show that an arbitrary intersection of closed sets in $\mathbb{R}$ is a closed set in $\mathbb{R}$.
\begin{proof}

Let $\{A_i \subset \mathbb{R}, i \in I\}$ be an arbitrary collection of closed sets.   Let $\bigcap\limits_{i \in I} A_i$ be the intersection of the closed sets.  If this intersection is $\emptyset$ we are done.  Supposing it is not, we'll continue.

By\marginnote{DeMorgan's Law} definition $\mathbb{R} \setminus \bigcap\limits_{i \in I} A_i = \bigcup\limits_{i \in I} (\mathbb{R} \setminus A_i)$.

Since $\{A_i\} \subset \mathbb{R}$ are closed, $(\mathbb{R} \setminus A_i) \subset \mathbb{R}$ is made up of open sets.  So we have an arbitrary union of open sets in $\mathbb{R}$ which we have already shown to be open.  This means  $\mathbb{R} \setminus \bigcap\limits_{i \in I} A_i$ is also open.  Therefore it's compliment $\bigcap\limits_{i \in I}A_i$ is closed.

\end{proof}

\item Show that a finite union of closed sets in $\mathbb{R}$ is a closed set in $\mathbb{R}$ 
\begin{proof}

Let $\{A_i \subset \mathbb{R}, i=1, 2,...,n\}$ be a finite collection of some $n \in \mathbb{N}$ closed sets in $\mathbb{R}$.  Let $\bigcup\limits_{i=1}^{n}A_i$ be the union of the finite closed subsets.

Summoning DeMorgan's Law, once more
\[\mathbb{R} \setminus \bigcup\limits_{i=1}^{n} A_i = \bigcap\limits_{i=1}^{n}(\mathbb{R} \setminus A_i)\]

Since $\{A_i\} \subset \mathbb{R}$ are closed, $(\mathbb{R} \setminus A_i) \subset \mathbb{R}$ is made up of open sets.  This means we have a a finite intersection of open sets, which we have already shown to be open.  Therefore $\mathbb{R} \setminus \bigcup\limits_{i=1}^{n} A_i $ is also open.  Which allows us to conclude, the complement,  $\bigcup\limits_{i=1}^{n} A_i $ must be closed.

\end{proof}

\item Show, by example, that an infinite union of closed sets in $\mathbb{R}$ is not necessarily closed in $\mathbb{R}$

One example is $\bigcup\limits_{n=1}^{\infty}[\frac{1}{n}, n]$  The union of this is open because you will never find a point of the union that lives at the boundary of the union.

Also, consider that $\mathbb{R}$ can be constructed of an infinite union of closed sets.   For example, each one point set of one point in $\mathbb{R}$, however we know that $\mathbb{R}$ is not closed.
\end{enumerate}

\subsection{Exercise 1.6.36}
Show that a subset of $\mathbb{R}$ is closed if and only if it contains all its accumulation points.

\begin{proof}"$\Rightarrow$" (Given in class)

Suppose $S$ is closed and $x$ is an accumulation point.  We prove by contradiction.  Suppose $x \notin S$ then $x \in S^c$  This means, $\exists x$ such that
\[(x - \epsilon, x+\epsilon) \subset S^c \]
That is to say $(x-\epsilon, x+\epsilon) \cap S = \emptyset$.  Which contradicts that x is an accumulation point.
\end{proof}

\begin{proof}"$\Leftarrow$"
Suppose $S$ contains all it's limits points and let $S$ be open, prove by contradiction.  Therefore, $\exists x \in S^c$ such that $(x- \epsilon, x+\epsilon)$ contains atleast one element of $S$, fix this value to $\bar{x} \in S^c$.  Symbolically written as
\[(\bar{x}- \epsilon, \bar{x}+\epsilon) \cap S \neq \emptyset \text{,  } \forall \epsilon > 0\]

For all $n \in \mathbb{N}$ let $x_n \in (\bar{x}, \frac{1}{n}) \cap S$ 

Notice, ${x_n}$ is a sequence in $S$ that converges to $\bar{x} \notin S$, meaning that $\bar{x}$ is an accumulation point of $S$ that is not contained in the open $S$, so $S$ must be closed.
\end{proof}

\subsection{Exercise 1.6.37}

\begin{enumerate}

\item The Cantor set is closed.

Let $\mathcal{C} = \bigcap\limits_{n \in \mathbb{N}}C_n$ Where $C_n$ is finite union union of closed sets.  As, shown above the arbitrary intersection of closed sets, this is closed.

\item The Cantor set consist of all  numbers in the closed $[0,1]$ whose ternary expansion consists of only $0$s and $2$s and may end infinitely in $2$s.

In every stage of constructing the Cantor set, the middle third of the of the set is removed, i.e. the numbers whose ternary decimal expansion notation has a $1$ in the $nth$ decimal place.

For example in the first stage, all numbers with $1$ in the first decimal place are removed, in the second all numbers $1$ in the second decimal place and so on.

\item The Cantor set is uncountable

\begin{proof}
Suppose $\mathcal{C}$ is countable.
\[\mathcal{C} = \{x_1^1, x1^2, x1^2, x1^3, ..., x_1^n\}\]
for each $x$ write
\[x^i = 0.d_1^i,0.d_2^i,0.d_3^i,0.d_4^i,...,0.d_m^i \]

Let $(d_1, d_2, d_3, d_4,...d_m)$ be the sequence that differs from the diagonal sequence of $d^i= d_1^i, d_2^i, d_3^i, d_4^i,...,d_m^i$ such that
\[d_m = \{_{0 \text{ if } d^m_m = 2}^{2 \text{ if } d^m_m = 0} \}\]
The ternary expansion $d_m$, does not appear in the list $x^i$ since $d_m \neq d_m^m$.  So, $x=0.d_1d_2d_3d_4...d_m$ is in $\mathcal{C}$ but no element in $\mathcal{C}$ has two different ternary expansions using only $0$s and $2$s. So x doesn't appear in the list above, which is a contradiction, so $\mathcal{C}$ is uncountable.

\end{proof}


\item Every point in the Cantor set is an accumulation point of the Cantor set.

Every point of the Cantor set can be captured in an arbitrary union of closed sets.  So for any $x \in \mathcal{C}$, you can write $x$ in ternary as a limit of other numbers that have $0$s or $2$s in there ternary expansion. 

\begin{proof}
Let $C_n = \bigcup S_m$ where $S_m$ is some number $m$ of closed sets. For any $x \in S_m$ let $\bar{x}$ be the right end point of $S_m$, unless $x = \bar{x}$ then let $\bar{x}$ be the left endpoint. By construction, each subinterval of $S_m$ has length $\frac{1}{3^n}$.  Which means $|x - x_m| < \frac{1}{3^n}$.

This implies that the sequence of $x_m$ converges to $x$ and such is an accumulation point.  Since we chose $x$ to be arbitrary, this applies to all $x \in \mathcal{C}$.  
\end{proof}

\item The compliment of the Cantor Set in $[0,1]$ is a dense subset of $[0,1]$

In layman's terms, we delete the all the middle intervals at every stage of constructing the Cantor set.  Which would mean the compliment is dense.

\begin{proof} Let, $\mathcal{C} = \bigcap\limits_{n \in \mathbb{N}}C_n \text{ where }C_n = \bigcup S_n$

The length of intervals in $S_n$ is $\frac{1}{3^n} = 3^{-n}$ Let $0 \leq a < b \leq 1$. Then interval $(a, b) \subset [0,1]$.  Fix $n \in \mathbb{N}$ such that $3^{-n} < b-a$ so that the length of every interval in $S_n$ is $3^{-n} < b-a$

Therefore the intervals of $S_n$ do note overlap any interval of length $(b-a)$.  Therefore no interval of length $b-a$ is contained in $C_n$.  Likewise, since $C_n \subset \mathcal{C}$, no interval of length $b-a$ is contained in $\mathcal{C}$.

Since lengths of $a,b$ was arbitrary, there are no open intervals in $\mathcal{C}$, which means $\mathcal{C}$ is nowhere dense and $\mathcal{C}^c$ is dense.
\end{proof}

\end{enumerate}

\subsection{Exercise 1.6.42}
Show that a compact subset of $\mathbb{R}$ is both closed and compact.

\begin{proof}
Suppose $A$ is a compact subset of $R$.  $A \subset \bigcup\limits_{k=1}^{\infty}U_k$, where $U_k$ are open sets.

By compactness of $A$, $\exists n \in \mathbb{N}$ such that $A \subset \bigcup\limits_{k=1}^{n}U_k$ Thus we can say $A$ is bounded.

Consider $A^c$ and define it as $X = \mathbb{R} \setminus A$ and take any $x \in X$. For every $a \in A$ there are open sets $U_a = (a-\epsilon,a+\epsilon)$ and $V_a = (x-\epsilon, x+\epsilon)$, for some $\epsilon > 0$ such that $U_a \cap V_a = \emptyset$.

The sets $\{U_a | a \in A\}$ form an open cover over $A$ and since $A$ is compact, there are finitely many points, $m \in \mathbb{N}$ such that $A \subset \bigcup\limits_{j=1}^{m}U_{a_{j}}$, denote this as $U_A$ and $V_A=\bigcap\limits_{j=0}^{m}V_{a_{j}}$.  Then $U_A$ and $V_A$ are open and $U_A \cap V_A = \emptyset$.

Notice that $V_A \subset X = A^c$ and since $x \in V_A$ and we chose $x$ to be arbitrary,  $A^c$ is open, making $A$ closed.

\end{proof}

\subsection{Exercise 1.6.46}

A subset of $\mathbb{R}$ is compact if and only if it is sequentially compact.

\begin{proof}"$\Rightarrow$" (Given in class)
Suppose S is compact. Consider any infinite subsequence $(X_n) \subset S$.  We want to show that $(X_n)$ has a convergent subsequence whose limit is in $S$.

We cover any point of $S$ by $(x-1, x+1)$.  This gives us an open cover.  Since $S$ is compact, there is a finite cover.  Therefore the sequence $(X_n)$ is covered by finitely many intervals of length $2$, hence there is an interval with infinitely many points in $(x_n)$.  Next, apply the theorem that every bounded sequence has a convergent subsequence.  

We still need to show that the subsequence converges to a point $x^*$ in $S$. Since $x^*$ is an accumulation point and S is bounded and closed, $S$ is closed and therefore $x \in S$

\end{proof}

\begin{proof}"$\Leftarrow$"
Suppose $S$ is sequentially compact. To show that $S$ is also compact, let's assume $S$ is not.

Since $S$ is not compact, there is an infinite subsequence that is not bounded in $S$.  Since this subsequence is not bounded, it does not converge to a point in $S$ which contradicts that $S$ is sequentially compact and therefore $S$ is compact.
\end{proof}

\subsection{Exercise 1.9.6}
\begin{enumerate}

\item If $N \in \mathbb{N}$ and $z \neq 1$, show that $S_N = \sum_{n=0}^{N}z^n = \frac{1-z^{N+1}}{1-z}$

\begin{proof}By induction.
Base case. $N=0$.
\[z^0 = \frac{1-z^{0+1}}{1-z}\]
\[1 = 1\]
Induction step.  
\begin{align*}
\sum_{n=0}^{N}z^n &= z^0+z^1+...+z^{N-1} + z^{N}\\
 &= \frac{1-z^{(N-1)+1}}{1-z} + z^{N}\\
 &= \frac{1-z^{N} + (1-z)z^{N}}{1-z}\\
 &= \frac{1-z^{N} + z^{N} - z^{N+1}}{1-z}\\
 &= \frac{1-z^{N+1}}{1-z}\\
\end{align*}
\end{proof}

\item If $|z| < 1$, show that $\lim_{n \rightarrow \infty}z^n =0$

Let the sequence $(z_n)_{n \in \mathbb{N}}$ be defined as the sequence of numbers where $z_n = z^n$.  If $|z| < 0$ its easy to see that for any $\epsilon > 0$ there exists $N \in \mathbb{N}$ such that for any $n > N$ $|z_n| < \epsilon$ Which is to say the $\lim_{n \rightarrow \infty}z^n =0$

\item If $|z| > 1$, show that $\lim_{n \rightarrow \infty}z^n$ does not exist.

Defining $(z_n)_{n \in \mathbb{N}}$ as above, with the caveat that $|z| > 1$, this sequence is ever increasing, such that as $n$ gets larger the distance between $n$ and $n+1$ gets larger.  Therefore, the limit does not exist.

\end{enumerate}

\subsection{Exercise 1.9.16}

\begin{enumerate}

\item If $p \in \mathbb{R}$ and $p < 1$, show that $\sum_{n=1}^{\infty}\frac{1}{n^p}$ diverges.

A key idea hear is the contrapositive of the Theorem $\sum_{n=1}^{\infty}a_n$ converges then $\lim_{n \rightarrow \infty}a_n = 0$. The contrapositive can be worded to say, if the terms of the series $a_n$ do not go to zero, then the series must diverge.

Then given any $\epsilon > 0$ there must be an item in the sequence that is makes the statement $|a_n| < \epsilon$ true.  

If  $p < 1$, we have a series $\sum_{n=1}^{\infty}\frac{1}{n^p}$ where our denominator $n^p$ will be ever increasing.  This is very similar to the harmonic series $\sum_{n=1}^{\infty}\frac{1}{n}$ which we know is divergent.  If $h_n = \frac{1}{n}$ and $a_n = \frac{1}{n^p}$, when $p < 1$, it's easy to see that
$\frac{1}{n} \leq \frac{1}{n^p}$, so the series $\sum_{n=1}^{\infty}\frac{1}{n^p}$ diverges.


\item If $p \in \mathbb{R}$ and $p > 1$, show that $\sum_{n=1}^{\infty}\frac{1}{n^p}$ converges.
We are given the hint...
\[ \sum_{n=2}^{N}\frac{1}{n^p} \leq \int_{1}^{N} \frac{1}{x^p}dx = \frac{1}{p-1}(1-\frac{1}{N^{p-1}})\]


\begin{proof}


Let $R_N = \int_{1}^{N} \frac{1}{x^p}dx = \frac{1}{p-1}(1-\frac{1}{N^{p-1}})$

We  say that the series $\sum_{n=1}^{\infty}\frac{1}{n^p}$ converges if the sequence of partial sums $(S_N)_{N \in \mathbb{N}}$  The series converges if and only if a sequence is a Cauchy sequence.  Which is to say, given $\epsilon > 0$, there is a $N_\epsilon\in \mathbb{N}$ such that for $N,M > N_\epsilon$, (assuming $N>M$), then $|\sum_{k=m+1}^{n}a_n| < \epsilon$

\begin{align*}
|S_N - S_M| &< e \\
|S_N - S_M| \leq |(1 + R_N) - (1 + R_M)| &< \epsilon \\
|\frac{1}{p-1}(1-\frac{1}{N^{p-1}}) - \frac{1}{p-1}(1-\frac{1}{M^{p-1}})| &< \epsilon \\
|\frac{1}{p-1}(1-\frac{1}{N^{p-1}} - 1-\frac{1}{M^{p-1}})| &< \epsilon\\
|\frac{1}{p-1}(-\frac{1}{N^{p-1}} - \frac{1}{M^{p-1}})| &< \epsilon\\
\end{align*}

\end{proof}
\end{enumerate}

\end{document}
\grid
