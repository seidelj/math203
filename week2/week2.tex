\documentclass{tufte-book}

\usepackage{amsmath, amsthm}
\usepackage{graphicx}
\setkeys{Gin}{width=\linewidth,totalheight=\textheight,keepaspectratio}
\graphicspath{{graphics/}}

\title{Real Analysis\\Second Week }
\author{Joe Seidel}
\date{\today}

\usepackage{booktabs}
\usepackage{units}
\usepackage{fancyvrb}
\fvset{fontsize=\normalsize}
\usepackage{multicol}
\usepackage{lipsum}
\usepackage{pdfpages}
\usepackage{tikz}
\usepackage{wasysym}

\newcommand{\doccmd}[1]{\texttt{\textbackslash#1}}% command name -- adds backslash automatically
\newcommand{\docopt}[1]{\ensuremath{\langle}\textrm{\textit{#1}}\ensuremath{\rangle}}% optional command argument
\newcommand{\docarg}[1]{\textrm{\textit{#1}}}% (required) command argument
\newenvironment{docspec}{\begin{quote}\noindent}{\end{quote}}% command specification environment
\newcommand{\docenv}[1]{\textsf{#1}}% environment name
\newcommand{\docpkg}[1]{\texttt{#1}}% package name
\newcommand{\doccls}[1]{\texttt{#1}}% document class name
\newcommand{\docclsopt}[1]{\texttt{#1}}% document class option name


\newtheoremstyle{mytheoremstyle} % name
	{\topsep}		% Space above
	{\topsep}		% Space below
	{\itshape}		% Body font
	{}			% Indent amount
	{\bfseries}	% Theorem head font
	{\textnormal{:}}	% Punctuation after theorem head
	{.5em}		% Space after theorem head
	{}			%Theorem headspec 
\theoremstyle{mytheoremstyle}
\newtheorem*{thm}{Thm.}

\newtheoremstyle{mylemstyle} % name
	{\topsep}		% Space above
	{\topsep}		% Space below
	{\itshape}		% Body font
	{}			% Indent amount
	{\bfseries}	% Theorem head font
	{\textnormal{:}}	% Punctuation after theorem head
	{.5em}		% Space after theorem head
	{}			%Theorem headspec 
\theoremstyle{mylemstyle}
\newtheorem*{lem}{Lem.}


\newtheoremstyle{mydefstyle} % name
	{\topsep}		% Space above
	{\topsep}		% Space below
	{\normalfont}	% Body font
	{}			% Indent amount
	{\bfseries}	% Theorem head font
	{\textnormal{:}}	% Punctuation after theorem head
	{.5em}		% Space after theorem head
	{}			%Theorem headspec 
\theoremstyle{mydefstyle}
\newtheorem*{mydef}{Def.}
\newtheorem*{ex}{E.g.}

\begin{document}

\maketitle
\pagenumbering{gobble}
\newpage
\pagenumbering{arabic}

\subsection{Exercise 1.6.35}

\begin{enumerate}
\item Show the $\emptyset$ and $\mathbb{R}$ are the only subsets of $\mathbb{R}$ that are both open and closed.

\begin{proof}
Let S be a non-empty open and closed set of $\mathbb{R}$.  Fix $x_0 \in S$ and $S \neq \mathbb{R}$.  Then, $\exists y \in \mathbb{R} \setminus S$.  Without loss of generality we may assume $y > x_0$.

Therefore we can form a set
\[I = \{ x \in \mathbb{R} | x > x_0, x \notin S\}\]

By construction\marginnote{greatest lower bound} this set is bounded below by $x_0$ and not empty because $y > x_0$, therefore $y \in I$.  Therefore we can let $i = \inf I$

Suppose $i \in S$.  Since $S$ is open it contains the open interveral $(i-\epsilon, i+\epsilon)$ for $\epsilon > 0$.  However, this interval contradicts that $i = \inf I$ because it implies a sequence $i_n > i$, $|i - i_n| < \frac{1}{n}$, where $i_n \in I$, which means $i_n \notin S$.  Which is not possible because $[i, i+e) \subset S$

Now suppose $i \notin S$.  Since $S$ is closed, $S^c$ which means that it contains an open interval $(i-\epsilon, i+\epsilon)$, but this contradicts the definition $i = \inf I$ because then we can find $i - \frac{e}{2}$ that is in $S^c$.

Therefore $S = \mathbb{R}$.
\end{proof}

\item Show that every non-empty open set in $\mathbb{R}$ can be written as a countable union of pairwise disjoint open intervals.

\begin{proof}

Let $U \subset \mathbb{R}$.  Let $\mathcal{O}$ be the set of open intervals that are a subsection of $U$.  For $I,J \in \mathcal{O}$ define $I \sim J$ iff there are

\[I_0 = I, I_1, I_2, ..., I_n = J  \in \mathcal{O}\]

Such that $I_k \cap I_{k+1} \neq \emptyset$ for $k=0, ..., n-1$.  Then $\sim$ defines an equivalence relation on $\mathcal{O}$.  For $I \in \mathcal{O}$ let $[I]$ be the $\sim$ of $I$.  Then $\{\cup[I] \text{ for } I \in \mathcal{O}\}$ is decomposition of $U$ into pairwise disjoint intervals.  By construction, these intervals are countable.

\end{proof}

\item Show that an arbitrary union of open sets in $\mathbb{R}$ is open in $\mathbb{R}$.

\begin{proof}
Suppose $\{Aß_i \subset \mathbb{R} | i \in I\}$ is an arbitrary collection of open sets.

If $x \in \cup A_i$ then $x \in A_i$ for some $i \in I$.  Since $A_i$ is open,  there $(x - \epsilon, x+ \epsilon) \subset A_i$ for $\epsilon > 0$.  Therefore,
\[ (x-\epsilon, x+\epsilon) \subset \bigcup\limits_{i \in I}A_i \]

 since $A_i \subset \mathbb{R}$ we arrive at our conclusion: The arbitrary union of open sets, $\bigcup\limits_{i \in I}A_i$ is open in $\mathbb{R}$. 
\end{proof}

\item Show that a finite intersection of open sets in $\mathbb{R}$ is open in $\mathbb{R}$
\begin{proof}
Suppose $\{A_i \subset \mathbb{R} | i = 1,2,...n\}$ is a finite collection of open sets.  If $ x \in \bigcap\limits_{i=1}^{n} A_i $ then $x \in A_i$ for every $1 \leq i \leq n$ Since $A_i$ is open, there are $\epsilon_i > 0$  such that $(x - \epsilon_i , x+ \epsilon_i) \subset A_i$

Let $\epsilon = min(\epsilon_1,..., \epsilon_n) > 0$.  This shows $(x - \epsilon, x+ \epsilon) \subset  \bigcap\limits_{i=1}^{n} A_i $

\end{proof}

\item Show, by example, that an infinite intersection of open sets is not necessarily open.
The open interval $(- \frac{1}{n},\frac{1}{n})$ is open $\forall n \in \mathbb{N}$  However, it's intersection
\[ \bigcap\limits_{n=1}^{\infty}(- \frac{1}{n},\frac{1}{n}) = {0} = [0..0] \]
which is closed.  I

\item Show that an arbitrary intersection of closed sets in $\mathbb{R}$ is a closed set in $\mathbb{R}$.
\begin{proof}

Let $\{A_i \subset \mathbb{R}, i \in I\}$ be an arbitrary collection of closed sets.   Let $\bigcap\limits_{i \in I} A_i$ be the intersection of the closed sets.  If this intersection is $\emptyset$ we are done.  Supposing it is not, we'll continue.

By\marginnote{DeMorgan's Law} definition $\mathbb{R} \setminus \bigcap\limits_{i \in I} A_i = \bigcup\limits_{i \in I} (\mathbb{R} \setminus A_i)$.

Since $\{A_i\} \subset \mathbb{R}$ are closed, $(\mathbb{R} \setminus A_i) \subset \mathbb{R}$ is made up of open sets.  So we have an arbitrary union of open sets in $\mathbb{R}$ which we have already shown to be open.  This means  $\mathbb{R} \setminus \bigcap\limits_{i \in I} A_i$ is also open.  Therefore it's compliment $\bigcap\limits_{i \in I}A_i$ is closed.

\end{proof}

\item Show that a finite union of closed sets in $\mathbb{R}$ is a closed set in $\mathbb{R}$ 
\begin{proof}

Let $\{A_i \subset \mathbb{R}, i=1, 2,...,n\}$ be a finite collection of some $n \in \mathbb{N}$ closed sets in $\mathbb{R}$.  Let $\bigcup\limits_{i=1}^{n}A_i$ be the union of the finite closed subsets.

Summoning DeMorgan's Law, once more
\[\mathbb{R} \setminus \bigcup\limits_{i=1}^{n} A_i = \bigcap\limits_{i=1}^{n}(\mathbb{R} \setminus A_i)\]

Since $\{A_i\} \subset \mathbb{R}$ are closed, $(\mathbb{R} \setminus A_i) \subset \mathbb{R}$ is made up of open sets.  This means we have a a finite intersection of open sets, which we have already shown to be open.  Therefore $\mathbb{R} \setminus \bigcup\limits_{i=1}^{n} A_i $ is also open.  Which allows us to conclude, the complement,  $\bigcup\limits_{i=1}^{n} A_i $ must be closed.

\end{proof}

\item Show, by example, that an infinite union of closed sets in $\mathbb{R}$ is not necessarily closed in $\mathbb{R}$

One example is $\bigcup\limits_{n=1}^{\infty}[\frac{1}{n}, n]$  The union of this is open because you will never find a point of the union that lives at the boundary of the union.

Also, consider that $\mathbb{R}$ can be constructed of an infinite union of closed sets.   For example, each one point set of one point in $\mathbb{R}$, however we know that $\mathbb{R}$ is not closed.
\end{enumerate}

\subsection{Exercise 1.6.36}
Show that a subset of $\mathbb{R}$ is closed iff it contains all its accumulation points.

\begin{proof}"$\Rightarrow$" (Given in class)

Suppose $S$ is closed and $x$ is an accumulation point.  We prove by contradiction.  Suppose $x \notin S$ then $x \in S^c$  This means, $\exists x$ such that
\[(x - \epsilon, x+\epsilon) \subset S^c \]
That is to say $(x-\epsilon, x+\epsilon) \cap S = \emptyset$.  Which contradicts that x is an accumulation point.
\end{proof}

\begin{proof}"$\Leftarrow$"
Suppose $S$ contains all it's limits points and let $S$ be open, prove by contradiction.  Therefore, $\exists x \in S^c$ such that $(x- \epsilon, x+\epsilon)$ contains atleast one element of $S$.  Symbolically written as
\[(x- \epsilon, x+\epsilon) \cap S \neq \emptyset \text{,  } \forall \epsilon > 0\]

For all $n \in \mathbb{N}$ let $x_n \in (x, \frac{1}{n}) \cap S$ 

Notice, ${x_n}$ is a sequence in $S$ that converges to $x \notin S$, meaning that $x$ is an accumulation point of $S$ that is not contained in the open $S$, so $S$ must be closed.
\end{proof}

\sub

\subsection{Exercise 1.6.42}
Show that a compact subset of $\mathbb{R}$ is both closed and compact.

\begin{proof}
Suppose $A$ is a compact subset of $R$.  $A \subset \bigcup\limits_{k=1}^{\infty}U_k$, where $U_k$ are open sets.

By compactness of $A$, $\exists n \in \mathbb{N}$ such that $A \subset \bigcup\limits_{k=1}^{n}U_k$ Thus we can say $A$ is bounded.

Consider $A^c$ and define it as $X = \mathbb{R} \setminus A$ and take any $x \in X$. For every $a \in A$ there are open sets $U_a = (a-\epsilon,a+\epsilon)$ and $V_a = (x-\epsilon, x+\epsilon)$, for some $\epsilon > 0$ such that $U_a \cap V_a = \emptyset$.

The sets $\{U_a | a \in A\}$ form an open cover over $A$ and since $A$ is compact, there are finitely many points, $m \in \mathbb{N}$ such that $A \subset \bigcup\limits_{j=1}^{m}U_{a_{j}}$, denote this as $U_A$ and $V_A=\bigcap\limits_{j=0}^{m}V_{a_{j}}$.  Then $U_A$ and $V_A$ are open and $U_A \cap V_A = \emptyset$.

Notice that $V_A \subset X = A^c$ and since $x \in V_A$ and we chose $x$ to be arbitrary,  $A^c$ is open, making $A$ closed.

\end{proof}
\end{document}
\grid
