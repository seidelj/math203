\documentclass{tufte-book}

\usepackage{amsmath, amsthm}
\usepackage{graphicx}
\setkeys{Gin}{width=\linewidth,totalheight=\textheight,keepaspectratio}
\graphicspath{{graphics/}}

\title{Real Analysis\\Fourth Week }
\author{Joe Seidel}
\date{\today}

\usepackage{booktabs}
\usepackage{units}
\usepackage{fancyvrb}
\fvset{fontsize=\normalsize}
\usepackage{multicol}
\usepackage{lipsum}
\usepackage{pdfpages}
\usepackage{tikz}
\usepackage{wasysym}

\newcommand{\doccmd}[1]{\texttt{\textbackslash#1}}% command name -- adds backslash automatically
\newcommand{\docopt}[1]{\ensuremath{\langle}\textrm{\textit{#1}}\ensuremath{\rangle}}% optional command argument
\newcommand{\docarg}[1]{\textrm{\textit{#1}}}% (required) command argument
\newenvironment{docspec}{\begin{quote}\noindent}{\end{quote}}% command specification environment
\newcommand{\docenv}[1]{\textsf{#1}}% environment name
\newcommand{\docpkg}[1]{\texttt{#1}}% package name
\newcommand{\doccls}[1]{\texttt{#1}}% document class name
\newcommand{\docclsopt}[1]{\texttt{#1}}% document class option name
\DeclareMathOperator{\proj}{proj}
\newcommand{\vct}{\mathbf}


\newcommand{\dprod}[2]{\langle #1, #2 \rangle}

\newtheoremstyle{mytheoremstyle} % name
	{\topsep}		% Space above
	{\topsep}		% Space below
	{\itshape}		% Body font
	{}			% Indent amount
	{\bfseries}	% Theorem head font
	{\textnormal{:}}	% Punctuation after theorem head
	{.5em}		% Space after theorem head
	{}			%Theorem headspec 
\theoremstyle{mytheoremstyle}
\newtheorem*{thm}{Thm.}

\newtheoremstyle{mylemstyle} % name
	{\topsep}		% Space above
	{\topsep}		% Space below
	{\itshape}		% Body font
	{}			% Indent amount
	{\bfseries}	% Theorem head font
	{\textnormal{:}}	% Punctuation after theorem head
	{.5em}		% Space after theorem head
	{}			%Theorem headspec 
\theoremstyle{mylemstyle}
\newtheorem*{lem}{Lem.}


\newtheoremstyle{mydefstyle} % name
	{\topsep}		% Space above
	{\topsep}		% Space below
	{\normalfont}	% Body font
	{}			% Indent amount
	{\bfseries}	% Theorem head font
	{\textnormal{:}}	% Punctuation after theorem head
	{.5em}		% Space after theorem head
	{}			%Theorem headspec 
\theoremstyle{mydefstyle}
\newtheorem*{mydef}{Def.}
\newtheorem*{ex}{E.g.}

\begin{document}

\maketitle
\pagenumbering{gobble}
\newpage
\pagenumbering{arabic}

\subsection{Exercise 3.2.5}
Let $X$ be any non-empty set and, for $x_1, x_2 \in X$, define
\[d(x_1, x_2) = 
\begin{cases}
      0, & \text{if}\ x_1=x_2 \\
      1, & \text{if}\ x_1\neq x_2\\
\end{cases}
\]
Show that $d$ is a metric on $X$. This is called the \textit{discrete metric}, the pair $(X,d)$ is referred to as a \textit{discrete metric space}.

In order to be a metric, need to show positive definiteness, symmetry and triangle inequality.

\begin{enumerate}
\item For $x_1, x_2 \in X$, $d(x_1, x_2) \geq 0$, and $d(x_1, x_2) = 0$ if and only if $x_1=x_2$.\\
This is trivial considering the given case.  If $x_1 \neq x_2$ $d(x_1,x_2) = 1 \geq 0$. If $x_1 = x_2$ then $d(x_1, x_2) = 0$.  Likewise, if $d(x_1, x_2) = 0$ then $x_1 = x_2$.

\item For any $x_1, x_2 \in X$, we have $d(x_1, x_2) = d(x_2, x_1)$.\\
Consider $x_1 = x_2$, $d(x_1, x_2) = 0$ and $d(x_2, x_1) = 0$ therefore $d(x_1, x_2) = d(x_2, x_1)$.  Consider $x_1 \neq x_2$, $d(x_1, x_2) = 1$ and $d(x_2, x_1) = 1$ therefore $d(x_1, x_2) = d(x_2, x_1)$.

\item For any $x_1, x_2, x_3 \in X$, we have 
\[d(x_1,x_2) \leq d(x_1, x_3) + d(x_3, x_2) \]

Since $d(x_i, x_j)$ is equal to either $1$ or $0$ for $i \neq j$.\\
Case 1, choose $x_1, x_2, x_3 \in X$. let $x_1 = x_2$ and let $x_3$ be arbitrary. Then $d(x_1,x_2) = 0$  and $d(x_1,x_3) + d(x_3,x_2) \geq 0$. Satisfying the triangle inequality.\\
Case 2, choose $x_1, x_2, x_3 \in X$, let $x_1 \neq x_2$.  Then $d(x_1,x_2) = 1$  Choose $x_3$ such that $d(x_1, x_3) = 0$ and $d(x_2,x_3) = 0$  This implies $x_1=x_3$ and $x_2=x_3$ which means $x_1=x_2$ which is a contradiction.  So atleast one of $d(x_1, x_3) = 1$ or $d(x_3, x_2) = 1$.  Satisfying the triangle inequality.
\end{enumerate}

\subsection{Exercise 3.2.6} (NOT ASSIGNED)\\
Let $(X,d)$ be a metric space, and let $Y$ be a proper subset of $X$.  Show that $(Y,d')$ is a metric space, where we define $d'(y_1,y_2) = d(y_1,y_2)$.  We call $d'$ the \textit{inherited metric} on $Y$.\\

Keeping in mind that $Y$ is a proper subset of $X$.
\begin{enumerate}

\item Consider $d'(y_1, y_2)$, since $(X,d)$ is a metric space $d(y_1, y_2) \geq 0$ and equal to zero if and only if $y_1 = y_2$, then by definition of $d'(y_1, y_2) = d(y_1,y_2)$ the same positive definiteness holds for $(Y,d')$.

\item Since since $(X,d)$ is a metric space $d(y_1, y_2) = d(y_2, y_1)$ which implies $d(y_2, y_1) = d'(y_2, y_1)$  Therefore  $d'(y_1, y_2) = d'(y_2, y_1)$.

\item Consider $d'(y_1, y_2) = d(y_1,y_2)$, $d'(y_1, y_3) = d(y_1,y_3)$ and $d'(y_3, y_2) = d(y_3,y_2)$. Then $d(y_1, y_2) \leq d(y_1, y_3) + d(y_3, y_2)$ implies $d'(y_1, y_2) \leq d'(y_1, y_3) + d'(y_3, y_2)$ satisfying the triangle inequality.

\end{enumerate}

\subsection{Exercise 3.2.8}
(NOT ASSIGNED)\\
Prove that $d_p$ is a metric on $\mathbb{R}^n$ for $p>1$. \textit{Hint:} The triangle inequality of the only hard part.  The proof depends on on H{\"o}lder's Inequality.  To begin, observe that
\[\|x + y\|_p^p - \sum_i|x_i+y_i|^p \leq \sum_i|x_i+y_i|^{p-1}|x_i| +\sum_i|x_i+y_i|^{p-1}|y_i| \]

\begin{proof}
To prove the triangle inequality, apply H{\"o}lder's Inequality using $q = \frac{p}{p-1}$.
\begin{align*}
\|x+y\|_p^p &\leq (\sum_i|x_i+y_i|^{p-1})^{\frac{p}{p-1}})^{\frac{p-1}{p}}(\sum_i|x_i|^p)^{\frac{1}{p}}\\
&+ (\sum_i|x_i+y_i|^{p-1})^{\frac{p}{p-1}})^{\frac{p-1}{p}}(\sum_i|y_i|^p)^{\frac{1}{p}}\\
&= \|x\|_p(\sum_i|x_i+y_i|^p)^{\frac{p-1}{p}} + \|y\|_p(\sum_i|x_i+y_i|^p)^{\frac{p-1}{p}}\\
&= \|x\|_p \|x+y\|_p^{p-1} +\|y\|_p \|x+y\|_p^{p-1}\\
&= (\|x\|_p + \|y\|_p) \|x+y\|_p^{p-1}
\end{align*}

We can divide both sides by $\|x+y\|_p^{p-1}$ to get
\[\|x+y\|_p \leq \|x\|_p + \|y\|_p \]
\end{proof}

\subsection{Exercise 3.2.9}
Note that H{\"o}lder's Inequality only works for $p,q > 1$.  Prove the triangle inequality for the $d_1$ metric.

\begin{proof}
In the $d_1$ metric, $d(x,y) = \|x-y\|_p$
\begin{align*}
 \|x+y\|_1 = \sum_i|x_i+y_1|^1 &\leq \sum_i|x_i|^1 + \sum|y_i|^1 \\
 &= \|x\|_1 + \|x\|_1
\end{align*}
\end{proof}

\subsection{Exercise 3.2.10}
Prove that $d_\infty$ defines a metric on $\mathbb{R}^n$
\begin{proof}
Need to show positive definiteness, symmetry, and triangle inequality.
\begin{enumerate}
\item Positive definiteness. $d_\infty(x,y) = max_{1\leq j\leq n}|x_j-y_j| \geq 0$ and $0$ if and only if $x = y$.

\item $d_\infty(x,y) = max_{1\leq j\leq n}|x_j-y_j| =  max_{1\leq j\leq n}|y_j-x_j| =  d_\infty(y,x)$

\item Triangle inequality.   Let $X \subset \mathbb{R}^n$ and consider $x, y \in X$.
\begin{align*}
\|x+y\|_\infty &= max_{1\leq j\leq n}|x_i+y_j| \\
&\leq max_{1\leq j\leq n}|x_j| + max_{1\leq j\leq n}|y_j| = \|x\|_\infty + \|y\|_\infty
\end{align*}
\end{enumerate}

\end{proof}

\subsection{Exercise 3.3.5}
If $1 \leq p < q$, show that the unit ball in $\ell_n^p(\mathbb{R})$ is contained in the unit ball in $\ell_n^q(\mathbb{R})$.
\begin{proof}
\marginnote{if $0 < a \leq 1$ then $(\sum_i|x_i|)^a \leq \sum_i|x_i|^a$ }Suppose $1 \leq p < q$
\begin{align*}
\|x\|_q = (\sum_i|x_i|^q)^{\frac{1}{q}} &= (\sum_i|x_i|^q)^{\frac{p}{qp}} \\
&\leq (\sum_i|x_i|^{q\frac{p}{q}})^{\frac{1}{p}} = (\sum_i|x_i|^p)^{\frac{1}{p}} = \|x\|_p
\end{align*}

Finally, supposed $\|x\|_p < 1$.  This implies $\|x\|_q \leq \|x\|_p < 1$ which means any point of $\|x\|_p$ is contained in $\|x\|_q$, which implies $\ell_n^p \subset \ell^q_n$. 

\end{proof}

\subsection{Exercise 3.3.6}
Choose $p$ with $1 \leq p \leq \infty$, and let $\epsilon >0$.  Show that $B_\epsilon(0) = \{\epsilon \cdot x | x \in B_1(0)\}$. 

Let $(X,d)$ be a metric space.  Define $B_1(0)=\{x \in X | d(x, 0) < 1 \}$ and $B_\epsilon(0)=\{x \in X | d(x, 0) < \epsilon \}$.  If we take any point from $B_1(0)$ we have the inequality $d(x,0) < 1$.  Therefore, we can multiply both sides of the inequality by $\epsilon$ with the result $\epsilon \cdot d(x, 0) < \epsilon$ for all $x \in B_1(0)$.  Which defines $B_\epsilon(0)$.

\subsection{Exercise 3.3.7}
Consider a point $x \in \mathbb{R}^2$ that lies outside the unit ball in $\ell_2^1(\mathbb{R})$ and inside the unit ball in $\ell_2^\infty$.  Is there a $p$ between $1$ and $\infty$ such that $\|x\|_p=1$?  Do the same problem for $\mathbb{R}^n$.\\
For $\mathbb{R}^2$.
\begin{proof}
Choose $x = (x_1, x_2)$ $x \in \mathbb{R}^2$ such that $|x_1| \leq 1$ and $|x_2| \leq 1$ and $|x_1| + |x_2| > 1$ for $p=1$. Consider $f(p) = |x_1|^p + |x_2|^p$.  As $p$ approaches infinity $f(p)$  approaches $0$. Since $f(p)$ is a continuous function, there is some $p$ between one and $\infty$ where $f(p) = 1$ making $\|x\|_p = 1$.
\end{proof}
For $\mathbb{R}^n$.
\begin{proof}
Chose $x = (x_1, x_2,...,x_n)$, $x \in \mathbb{R}^n$ such that $|x_i| \leq 1$ and $|x_1| + |x_2| +...+|x_n| > 1$ for $p=1$.  Again consider $f(p) = |x_1|^p + |x_2|^p +...+|x_n|^p$.  As $p$ approaches infinity, $f(p)$ approaches $0$.  Again, since $f(p)$ is a continuous function, there is some $p$ between $1$ and $\infty$ such that $f(p) = 1$ making $\|x\|_p = 1$.
\end{proof}

\subsection{Exercise 3.3.10}
Prove that the following are open sets.

\begin{enumerate}
\item The "first quadrant", that is, $\{(x,y) \in \mathbb{R}^2 | x > 0 \text{ and } y > 0 \}$, in the usual metric.

\begin{proof}
Choose $x, y \in \mathbb{R}^2$ such that $x > 0$ and $y > 0$ and otherwise let them be arbitrary.    Since we know $\mathbb{R}$ is open, For $x$ choose $\epsilon_x > 0$ such that $(x-\epsilon_x, x+\epsilon_x) \subset \mathbb{R}_+$ and choose $\epsilon_y >0$ such that $(y-\epsilon_y, y+\epsilon_y) \subset \mathbb{R}_+$   Choose $\epsilon = min(\epsilon_x, \epsilon_y)$, then we have $B_\epsilon(x,y) \subset \mathbb{R}_+^2$.  Since $x,y$ only need to be positive, we can draw an open ball around any point $(x,y)$ in $\mathbb{R}_+^2$ making the first quadrant open.
\end{proof}

\item any subset of a discrete metric.

\begin{proof}
Let $(X, d)$ be a discrete metric.  Recall that in a discrete metric $d(x,x) = 0$ and $d(x,y)=1$, $x \neq y$.   You can take $0<r<1$ such that the only point the ball contains is the point that it is centered on.  If $r>1$ then it contains all the points.  In either case $B_r(x) \subset X$.
\end{proof}
\end{enumerate}

\subsection{Exercise 3.3.12}
Let $X = [-1,1]$ with the metric inherited above ($\mathbb{R}$).  Describe the open balls of $B_r(1)$ for various values of $r$.

\begin{enumerate}

\item Consider $0< r \leq 2$.  $B_r(1) = (-r, 1]$
\item Consider $r > 2$. $B_r(1) = [-1,1]$ 
\end{enumerate}
\end{document}
\grid
