\documentclass{tufte-book}

\usepackage{amsmath, amsthm}
\usepackage{graphicx}
\setkeys{Gin}{width=\linewidth,totalheight=\textheight,keepaspectratio}
\graphicspath{{graphics/}}

\title{Real Analysis\\Third Week }
\author{Joe Seidel}
\date{\today}

\usepackage{booktabs}
\usepackage{units}
\usepackage{fancyvrb}
\fvset{fontsize=\normalsize}
\usepackage{multicol}
\usepackage{lipsum}
\usepackage{pdfpages}
\usepackage{tikz}
\usepackage{wasysym}

\newcommand{\doccmd}[1]{\texttt{\textbackslash#1}}% command name -- adds backslash automatically
\newcommand{\docopt}[1]{\ensuremath{\langle}\textrm{\textit{#1}}\ensuremath{\rangle}}% optional command argument
\newcommand{\docarg}[1]{\textrm{\textit{#1}}}% (required) command argument
\newenvironment{docspec}{\begin{quote}\noindent}{\end{quote}}% command specification environment
\newcommand{\docenv}[1]{\textsf{#1}}% environment name
\newcommand{\docpkg}[1]{\texttt{#1}}% package name
\newcommand{\doccls}[1]{\texttt{#1}}% document class name
\newcommand{\docclsopt}[1]{\texttt{#1}}% document class option name


\newtheoremstyle{mytheoremstyle} % name
	{\topsep}		% Space above
	{\topsep}		% Space below
	{\itshape}		% Body font
	{}			% Indent amount
	{\bfseries}	% Theorem head font
	{\textnormal{:}}	% Punctuation after theorem head
	{.5em}		% Space after theorem head
	{}			%Theorem headspec 
\theoremstyle{mytheoremstyle}
\newtheorem*{thm}{Thm.}

\newtheoremstyle{mylemstyle} % name
	{\topsep}		% Space above
	{\topsep}		% Space below
	{\itshape}		% Body font
	{}			% Indent amount
	{\bfseries}	% Theorem head font
	{\textnormal{:}}	% Punctuation after theorem head
	{.5em}		% Space after theorem head
	{}			%Theorem headspec 
\theoremstyle{mylemstyle}
\newtheorem*{lem}{Lem.}


\newtheoremstyle{mydefstyle} % name
	{\topsep}		% Space above
	{\topsep}		% Space below
	{\normalfont}	% Body font
	{}			% Indent amount
	{\bfseries}	% Theorem head font
	{\textnormal{:}}	% Punctuation after theorem head
	{.5em}		% Space after theorem head
	{}			%Theorem headspec 
\theoremstyle{mydefstyle}
\newtheorem*{mydef}{Def.}
\newtheorem*{ex}{E.g.}

\begin{document}

\maketitle
\pagenumbering{gobble}
\newpage
\pagenumbering{arabic}


\subsection{Exercise 1.9.8}
Give examples to show that if $r=1$ in the statement of the Ratio Test, anything may happen.


First consider the harmonic series, $\sum_{n=1}^{\infty}\frac{1}{n}$. The $\lim_{n \to \infty}|\frac{a_{n+1}}{a_n}| = 1$

\[ \lim_{n \to \infty}|\frac{\frac{1}{n+1}}{\frac{1}{n}}| = \lim_{n \to \infty}|\frac{n}{n+1}| = 1 \]

However, we know that the series $\sum_{n=1}^{\infty}\frac{1}{n}$ diverges.

Next consider $\sum_{n=1}^{\infty}\frac{(-1)^{n+1}}{n}$.  Again $\lim_{n \to \infty}|\frac{a_{n+1}}{a_n}| = 1$

\begin{align*}
\lim_{n \to \infty}|\frac{\frac{(-1)^{n+2}}{n+1}}{\frac{(-1)^{n+1}}{n}}| &= \lim_{n \to \infty}|\frac{(-1)^{n+2}}{n+1} \frac{n}{(-1)^{n+1}}| \\
&= \lim_{n \to \infty}|\frac{-n}{n+1}|\\
&= 1
\end{align*}

This series, $\sum_{n=1}^{\infty}\frac{(-1)^{n+1}}{n}$ converges since $|\sum_{k=m+1}^{n}\frac{(-1)^{k+1}}{k}| < \frac{1}{m}$ but not absolutely because $\sum_{n=1}^{\infty}|\frac{(-1)^n}{n}|$ goes to infinity.

Finally, consider the series $\sum_{n=1}^{\infty} \frac{1}{n^2}$  This too has $\lim_{n \to \infty} |\frac{a_{n+1}}{a_n}| = 1$.

\begin{align*}
\lim_{n \to \infty} |\frac{\frac{1}{n+1^2}}{\frac{1}{n^2}}| &= \lim_{n \to \infty} |\frac{1}{(n+1)^2} \frac{n^2}{1}| \\
&= \lim_{n \to \infty}\frac{n^2}{(n+1)^2} \\
&= 1
\end{align*}

The series $\sum_{n=1}^{\infty} \frac{1}{n^2}$ is absolutely convergent because $\sum_{n=1}^{\infty}|\frac{1}{n^2}|$ converges. 

\subsection{Exercise 1.9.20}
Give examples to show that if $r = 1$ in the statement of the Root Test, anything my happen.

Borrowing from the ideas in the last exercise, consider the series $\sum_{n=1}^{\infty} \frac{1}{n^2}$ and $\sum_{n=1}^{\infty} \frac{1}{n}$.  Both have $r = 1$.

\[\limsup_{n \to \infty}|\frac{1}{n}|^{\frac{1}{n}} = 1 \]
\[\limsup_{n \to \infty}|\frac{1}{n^2}|^{\frac{1}{n}} =1 \] 

Anything may happen when $r = 1$

\subsection{Exercise 1.9.26}
Determine the radius of convergence of the following power series:

\[ r = \limsup_{n \to \infty}|a_n|^{\frac{1}{n}} \]

Where the radius of convergence is $\frac{1}{r}$ if $r > 0$, $\infty$ if $r = 0$ and $0$ if lim sup does not exist, $(r = \infty)$.

\begin{enumerate}

\item $\sum_{n=1}^{\infty}\frac{z^n}{n!}$

\[ \sum_{n=1}^{\infty}\frac{z^n}{n!} = \sum_{n=1}^{\infty}\frac{1}{n!}z^n \]
\[ \lim_{n \to \infty}|\frac{1}{n!}|^{\frac{1}{n}} = 0 \]

Radius of convergence is $\infty$

\item $\sum_{n=2}^{\infty}\frac{z^n}{\ln(n)}$

\[ \sum_{n=1}^{\infty}\frac{z^n}{\ln(n)} = \sum_{n=2}^{\infty}\frac{1}{\ln(n)}z^n \]
\[ \lim_{n \to \infty}|\frac{1}{\ln(n)}|^{\frac{1}{n}} = 1 \]

Radius of convergence is $1$.

\item $\sum_{n=1}^{\infty}\frac{n^n}{n!}z^n$
\marginnote{$e^n = \sum_{n=1}^{\infty}\frac{n^n}{n!}$} 
\[ \lim_{n \to \infty}|\frac{n^n}{n!}|^{\frac{1}{n}} = e \]

Radius of convergence is $\frac{1}{e}$
\end{enumerate}

\subsection{Exercise 2.5.8}
Prove that the scalar product is a positive definite symmetric bilinear form on $\mathbb{E}^n$.

\begin{proof}
The scalar product of vectors $\mathbf{v} = (v_1, v_2, v_3,...,v_n)$ and $\mathbf{w} = (w_1, w_2, w_3,...,w_n)$ is $\langle\mathbf{v}, \mathbf{w}\rangle = v_1w_1 + v_2w_2 + v_3w_3 +...+ v_nw_n$. 

Let $V = \mathbb{E}^n$ be a vector space over $F = \mathbb{R}$.  A bilinear form $\langle.,.\rangle$ on $V$ is a map 
\[\langle.,.\rangle : V \times V \rightarrow F \]
The satisfies linearity in both variables.  That is, for all $\mathbf{v},\mathbf{v}_1\mathbf{v}_2,\mathbf{w},\mathbf{w}_1,\mathbf{w}_2 \in V$ and all $\alpha \in F$

\begin{align*}
\langle\mathbf{v}_1+\mathbf{v}_2, \mathbf{w} \rangle &= (v_{1_{1}}+v_{2_{1}})w_1 + (v_{1_{2}}+v_{2_{2}})w_2 + (v_{1_{3}}+v_{2_{3}})w_3 +...+(v_{1_{n}}+v_{2_{n}})w_n  \\
&= v_{1_{1}}w_1 + v_{2_{1}}w1 + v_{1_{2}}w_2 + v_{2_{2}}w2 + v_{1_{3}}w_3 + v_{2_{3}}w3 +...+v_{1_{n}}w_n + v_{2_{n}}w_n \\
&= (v_{1_{1}}w_1 + v_{1_{2}}w_2 + v_{1_{3}}w_3 +...+v_{1_{n}}w_n) + (v_{2_{1}}w_1 + v_{2_{2}}w_2 + v_{2_{3}}w_3 +...+v_{2_{n}}w_n)\\
&= \langle\mathbf{v_1},\mathbf{w} \rangle + \langle\mathbf{v_2},\mathbf{w} \rangle
\end{align*}

\begin{align*}
\langle \alpha \mathbf{v},\mathbf{w} \rangle &= \alpha v_1 w_1 + \alpha v_2 w_2 + \alpha v_3 w_3 +...+ \alpha v_n w_n \\
&= \alpha(v_1w_1) + \alpha(v_2w_2) + \alpha(v_3w_3) +...+\alpha(v_nw_n) \\
&=\alpha (v_1w_1 + v_2w_2 + v_3w_3 +...+v_nw_n) \\
&= \alpha \langle \mathbf{v}, \mathbf{w} \rangle 
\end{align*}

\begin{align*}
\langle \mathbf{v}, \mathbf{w}_1+\mathbf{w}_2 \rangle &= v_1(w_{1_{1}}+w_{2_{1}}) + v_2(w_{1_{2}}+w_{2_{2}}) + v_3(w_{1_{3}}+w_{2_{3}}) +...+v_n(w_{1_{n}}+w_{2_{n}})  \\
&= v_1w_{1_{1}} + v_1w_{2_{1}} + v_2w_{1_{2}}+ v_2w_{2_{2}} + v_3w_{1_{3}}+ v_3w_{2_{3}} +...+v_nw_{1_{n}}+ v_nw_{2_{n}}\\
&= (v_1w_{1_{1}} + v_2w_{1_{2}} + v_3w_{1_{3}} +...+v_nw_{1_{n}}) +
(v_1w_{2_{1}} + v_2w_{2_{2}} + v_3w_{2_{3}} +...+v_nw_{2_{n}}) \\
&= \langle \mathbf{v}, \mathbf{w}_1 \rangle + \langle \mathbf{v}, \mathbf{w}_2 \rangle
\end{align*}

\begin{align*}
\langle \mathbf{v}, \alpha \mathbf{w} \rangle &= v_1 \alpha w_1 + v_2 \alpha w_2 + v_3 \alpha w_3 +...+ v_n \alpha w_n \\
&= \alpha(v_1w_1) + \alpha(v_2w_2) + \alpha(v_3w_3) +...+\alpha(v_nw_n) \\
&=\alpha (v_1w_1 + v_2w_2 + v_3w_3 +...+v_nw_n) \\
&= \alpha \langle \mathbf{v}, \mathbf{w} \rangle 
\end{align*}

Regarding being positive definite , consider $\langle\mathbf{v}, \mathbf{v}\rangle$
\begin{align*}
\langle\mathbf{v}, \mathbf{v}\rangle &= v_1v_1 + v_2v_2 + v_3v_3 +...+v_nv_n \\
&= v_1^2 + v_2^2 + v_3^2 +...+v_n^2\\
&\geq 0
\end{align*}

Notice also, that $\langle\mathbf{v}, \mathbf{v}\rangle = 0$ if and only if $\mathbf{v} = 0$

\end{proof}

\subsection{Exercise 2.5.11}
Prove the following properties of the norm if $\mathbf{v},\mathbf{w} \in \mathbb{E}^n$

\begin{enumerate}

\item $||\mathbf{v}|| \geq 0$

\begin{proof}
If $\mathbf{v} \in \mathbb{E}^n$ the norm is defined by
\begin{align*}
||\mathbf{v}|| &= \sqrt{\langle \mathbf{v}, \mathbf{v} \rangle}\\
&= \sqrt{(v_1v_1 + v_2v_2 + v_3v_3 +...+v_nv_n)} \\
&= \sqrt{(v_1^2 + v_2^2 + v_3^2 +...+v_n^2)} \\
&= (v_1^2 + v_2^2 + v_3^2 +...+v_n^2)^{\frac{1}{2}} \\
&\geq 0
\end{align*}
\end{proof}

\item $||\mathbf{v}|| = 0$ if and only if $\mathbf{v} = 0$

From the definition of positive definite, $\langle \mathbf{v}, \mathbf{v} \rangle = 0$ if and only if $\mathbf{v} = 0$ 
\begin{proof} "$\Rightarrow$"
Let $\|\mathbf{v}\| = 0$.  Then 
\[ \|\mathbf{v}\| = \sqrt{\langle \mathbf{v}, \mathbf{v} \rangle} = 0 \]
\begin{align*}
\sqrt{\langle \mathbf{v}, \mathbf{v} \rangle} &= 0\\
\sqrt{(v_1v_1 + v_2v_2 + v_3v_3 +...+v_nv_n)} &= 0\\
(v_1^2 + v_2^2 + v_3^2 +...+v_n^2)^{\frac{1}{2}} &= 0 \\
(v_1^2 + v_2^2 + v_3^2 +...+v_n^2)&= 0
\end{align*}
So $\mathbf{v} = 0$
\end{proof}

\begin{proof} "$\Leftarrow$"

Let $\mathbf{v} = 0$

So $\langle \mathbf{v}, \mathbf{v} \rangle = 0$. 

\begin{align*}
\|\mathbf{v}\| &= \sqrt{\langle \mathbf{v}, \mathbf{v} \rangle} \\ 
&= \sqrt{0} \\
&= 0
\end{align*}
\end{proof}

\item $\|\alpha \mathbf{v}\| = |\alpha| \|\mathbf{v}\|$, $\alpha \in \mathbb{r}$ 

\begin{proof}
\begin{align*}
\|\alpha \mathbf{v} \| &= \sqrt{\langle\alpha\mathbf{v},\alpha\mathbf{v}\rangle} \\
&= \sqrt{(\alpha^2v_1^2 +\alpha^2v_2^2 + \alpha^2v_3^2 +...+\alpha^2v_n^2)} \\
&= \sqrt{\alpha^2(v_1^2 + v_2^2 +v_3^2 +...+v_n^2)} \\
&= |\alpha| \sqrt{(v_1^2 + v_2^2 +v_3^2 +...+v_n^2)} \\
&= |\alpha| \sqrt{\langle \mathbf{v}, \mathbf{v} \rangle} \\
&= |\alpha| \|\mathbf{v}\|
\end{align*}
\end{proof}

\item $\|\mathbf{v} + \mathbf{w} \| \leq \|\mathbf{v}\| + \|\mathbf{w}\|$

\begin{proof}

First recall the Pythagorean theorem, $a^2 + b^2 = c^2$ and Cauchy-Schwarz Inequality, let $\mathbf{v}, \mathbf{w} \in \mathbb{E}^n$, then $|\langle \mathbf{v}, \mathbf{w} \rangle| \leq \|\mathbf{v}\|\|\mathbf{w}\|$
\begin{align*}
\|\mathbf{v} + \mathbf{w} \|^2 &= \langle \mathbf{v}+\mathbf{w},\mathbf{v}+\mathbf{w}\rangle \\
&= (v_1+w_1)^2 + (v_2+w_2)^2 +...+(v_n+w_n)^2\\
&= (v_1^2 + v_2^2 +...+v_n^2) + (w_1^2+w_2^2+...+w_n^2) + (v_1w_1+...+v_nw_n) +(w_1v_1+...+w_nv_n) \\ 
&= \langle \mathbf{v},\mathbf{v}\rangle + \langle \mathbf{v}, \mathbf{w} \rangle + \langle \mathbf{w}, \mathbf{v} \rangle + \langle \mathbf{w},\mathbf{w}\rangle \\
&\leq \|\mathbf{v}\|^2 + 2|\langle \mathbf{v}, \mathbf{w} \rangle| + \|\mathbf{w}\|^2 \\
&\leq \|\mathbf{v}\|^2 + 2\|\mathbf{v}\|\|\mathbf{w}\| + \|\mathbf{w}\|^2 \\
&= (\|\mathbf{v}\| + \|\mathbf{w}\|)^2
\end{align*}

\end{proof}

\item $\|\mathbf{v} + \mathbf{w} \|^2 + \|\mathbf{v} -\mathbf{w}\|^2 = 2(\|\mathbf{v}\|^2 + \|\mathbf{w}\|^2)$
\begin{proof}
\begin{align*}
\|\mathbf{v} + \mathbf{w} \|^2 &+ \|\mathbf{v} -\mathbf{w}\|^2 = 
  \langle \mathbf{v} + \mathbf{w},\mathbf{v} + \mathbf{w}\rangle +
  \langle \mathbf{v} - \mathbf{w},\mathbf{v} - \mathbf{w}\rangle \\
&= (v_1 +w_1)^2 + ... + (v_n+w_n)^2 + (v_1-w_1)^2 + ... + (v_n-w_n)^2\\
&= (v_1^2 + 2v_1w_1 + w_1^2 +...+v_n^2 + 2v_nw_n + w_n^2) + (v_1^2 - 2v_1w_1 + w_1^2 +...+v_n^2 - 2v_nw_n + w_n^2) \\
&= (2v_1^2 +...+2v_n^2) + (2w_1^2 +...+ 2w_n^2)\\
&= 2(v_1^2 +...+v_n^2) + 2(w_1^2 +...+w_n^2) \\
&= 2(\langle\mathbf{v},\mathbf{v}\rangle + \langle\mathbf{w},\mathbf{w}\rangle)\\
&= 2(\|\mathbf{v}\|^2 = \|\mathbf{w}\|^2)
\end{align*}
\end{proof}
\end{enumerate}
\end{document}
\grid