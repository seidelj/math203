\documentclass{tufte-book}

\usepackage{amsmath, amsthm}
\usepackage{graphicx}
\setkeys{Gin}{width=\linewidth,totalheight=\textheight,keepaspectratio}
\graphicspath{{graphics/}}

\title{Real Analysis\\Third Week }
\author{Joe Seidel}
\date{\today}

\usepackage{booktabs}
\usepackage{units}
\usepackage{fancyvrb}
\fvset{fontsize=\normalsize}
\usepackage{multicol}
\usepackage{lipsum}
\usepackage{pdfpages}
\usepackage{tikz}
\usepackage{wasysym}

\newcommand{\doccmd}[1]{\texttt{\textbackslash#1}}% command name -- adds backslash automatically
\newcommand{\docopt}[1]{\ensuremath{\langle}\textrm{\textit{#1}}\ensuremath{\rangle}}% optional command argument
\newcommand{\docarg}[1]{\textrm{\textit{#1}}}% (required) command argument
\newenvironment{docspec}{\begin{quote}\noindent}{\end{quote}}% command specification environment
\newcommand{\docenv}[1]{\textsf{#1}}% environment name
\newcommand{\docpkg}[1]{\texttt{#1}}% package name
\newcommand{\doccls}[1]{\texttt{#1}}% document class name
\newcommand{\docclsopt}[1]{\texttt{#1}}% document class option name
\DeclareMathOperator{\proj}{proj}
\newcommand{\vct}{\mathbf}


\newcommand{\dprod}[2]{\langle #1, #2 \rangle}

\newtheoremstyle{mytheoremstyle} % name
	{\topsep}		% Space above
	{\topsep}		% Space below
	{\itshape}		% Body font
	{}			% Indent amount
	{\bfseries}	% Theorem head font
	{\textnormal{:}}	% Punctuation after theorem head
	{.5em}		% Space after theorem head
	{}			%Theorem headspec 
\theoremstyle{mytheoremstyle}
\newtheorem*{thm}{Thm.}

\newtheoremstyle{mylemstyle} % name
	{\topsep}		% Space above
	{\topsep}		% Space below
	{\itshape}		% Body font
	{}			% Indent amount
	{\bfseries}	% Theorem head font
	{\textnormal{:}}	% Punctuation after theorem head
	{.5em}		% Space after theorem head
	{}			%Theorem headspec 
\theoremstyle{mylemstyle}
\newtheorem*{lem}{Lem.}


\newtheoremstyle{mydefstyle} % name
	{\topsep}		% Space above
	{\topsep}		% Space below
	{\normalfont}	% Body font
	{}			% Indent amount
	{\bfseries}	% Theorem head font
	{\textnormal{:}}	% Punctuation after theorem head
	{.5em}		% Space after theorem head
	{}			%Theorem headspec 
\theoremstyle{mydefstyle}
\newtheorem*{mydef}{Def.}
\newtheorem*{ex}{E.g.}

\begin{document}

\maketitle
\pagenumbering{gobble}
\newpage
\pagenumbering{arabic}


\subsection{Exercise 1.9.8}
Give examples to show that if $r=1$ in the statement of the Ratio Test, anything may happen.


First consider the harmonic series, $\sum_{n=1}^{\infty}\frac{1}{n}$. The $\lim_{n \to \infty}|\frac{a_{n+1}}{a_n}| = 1$

\[ \lim_{n \to \infty}|\frac{\frac{1}{n+1}}{\frac{1}{n}}| = \lim_{n \to \infty}|\frac{n}{n+1}| = 1 \]

However, we know that the series $\sum_{n=1}^{\infty}\frac{1}{n}$ diverges.

Next consider $\sum_{n=1}^{\infty}\frac{(-1)^{n+1}}{n}$.  Again $\lim_{n \to \infty}|\frac{a_{n+1}}{a_n}| = 1$

\begin{align*}
\lim_{n \to \infty}|\frac{\frac{(-1)^{n+2}}{n+1}}{\frac{(-1)^{n+1}}{n}}| &= \lim_{n \to \infty}|\frac{(-1)^{n+2}}{n+1} \frac{n}{(-1)^{n+1}}| \\
&= \lim_{n \to \infty}|\frac{-n}{n+1}|\\
&= 1
\end{align*}

This series, $\sum_{n=1}^{\infty}\frac{(-1)^{n+1}}{n}$ converges since $|\sum_{k=m+1}^{n}\frac{(-1)^{k+1}}{k}| < \frac{1}{m}$ but not absolutely because $\sum_{n=1}^{\infty}|\frac{(-1)^n}{n}|$ goes to infinity.

Finally, consider the series $\sum_{n=1}^{\infty} \frac{1}{n^2}$  This too has $\lim_{n \to \infty} |\frac{a_{n+1}}{a_n}| = 1$.

\begin{align*}
\lim_{n \to \infty} |\frac{\frac{1}{n+1^2}}{\frac{1}{n^2}}| &= \lim_{n \to \infty} |\frac{1}{(n+1)^2} \frac{n^2}{1}| \\
&= \lim_{n \to \infty}\frac{n^2}{(n+1)^2} \\
&= 1
\end{align*}

The series $\sum_{n=1}^{\infty} \frac{1}{n^2}$ is absolutely convergent because $\sum_{n=1}^{\infty}|\frac{1}{n^2}|$ converges. 

\subsection{Exercise 1.9.20}
Give examples to show that if $r = 1$ in the statement of the Root Test, anything my happen.

Borrowing from the ideas in the last exercise, consider the series $\sum_{n=1}^{\infty} \frac{1}{n^2}$ and $\sum_{n=1}^{\infty} \frac{1}{n}$.  Both have $r = 1$.

\[\limsup_{n \to \infty}|\frac{1}{n}|^{\frac{1}{n}} = 1 \]
\[\limsup_{n \to \infty}|\frac{1}{n^2}|^{\frac{1}{n}} =1 \] 

Anything may happen when $r = 1$

\subsection{Exercise 1.9.26}
Determine the radius of convergence of the following power series:

\[ r = \limsup_{n \to \infty}|a_n|^{\frac{1}{n}} \]

Where the radius of convergence is $\frac{1}{r}$ if $r > 0$, $\infty$ if $r = 0$ and $0$ if lim sup does not exist, $(r = \infty)$.

\begin{enumerate}

\item $\sum_{n=1}^{\infty}\frac{z^n}{n!}$

\[ \sum_{n=1}^{\infty}\frac{z^n}{n!} = \sum_{n=1}^{\infty}\frac{1}{n!}z^n \]
\[ \lim_{n \to \infty}|\frac{1}{n!}|^{\frac{1}{n}} = 0 \]

Radius of convergence is $\infty$

\item $\sum_{n=2}^{\infty}\frac{z^n}{\ln(n)}$

\[ \sum_{n=1}^{\infty}\frac{z^n}{\ln(n)} = \sum_{n=2}^{\infty}\frac{1}{\ln(n)}z^n \]
\[ \lim_{n \to \infty}|\frac{1}{\ln(n)}|^{\frac{1}{n}} = 1 \]

Radius of convergence is $1$.

\item $\sum_{n=1}^{\infty}\frac{n^n}{n!}z^n$
\marginnote{$e^n = \sum_{n=1}^{\infty}\frac{n^n}{n!}$} 
\[ \lim_{n \to \infty}|\frac{n^n}{n!}|^{\frac{1}{n}} = e \]

Radius of convergence is $\frac{1}{e}$
\end{enumerate}

\subsection{Exercise 2.5.8}
Prove that the scalar product is a positive definite symmetric bilinear form on $\mathbb{E}^n$.

\begin{proof}
The scalar product of vectors $\mathbf{v} = (v_1, v_2, v_3,...,v_n)$ and $\mathbf{w} = (w_1, w_2, w_3,...,w_n)$ is $\langle\mathbf{v}, \mathbf{w}\rangle = v_1w_1 + v_2w_2 + v_3w_3 +...+ v_nw_n$. 

Let $V = \mathbb{E}^n$ be a vector space over $F = \mathbb{R}$.  A bilinear form $\langle.,.\rangle$ on $V$ is a map 
\[\langle.,.\rangle : V \times V \rightarrow F \]
The satisfies linearity in both variables.  That is, for all $\mathbf{v},\mathbf{v}_1\mathbf{v}_2,\mathbf{w},\mathbf{w}_1,\mathbf{w}_2 \in V$ and all $\alpha \in F$

\begin{align*}
\langle\mathbf{v}_1+\mathbf{v}_2, \mathbf{w} \rangle &= (v_{1_{1}}+v_{2_{1}})w_1 + (v_{1_{2}}+v_{2_{2}})w_2 + (v_{1_{3}}+v_{2_{3}})w_3 +...+(v_{1_{n}}+v_{2_{n}})w_n  \\
&= v_{1_{1}}w_1 + v_{2_{1}}w1 + v_{1_{2}}w_2 + v_{2_{2}}w2 + v_{1_{3}}w_3 + v_{2_{3}}w3 +...+v_{1_{n}}w_n + v_{2_{n}}w_n \\
&= (v_{1_{1}}w_1 + v_{1_{2}}w_2 + v_{1_{3}}w_3 +...+v_{1_{n}}w_n) + (v_{2_{1}}w_1 + v_{2_{2}}w_2 + v_{2_{3}}w_3 +...+v_{2_{n}}w_n)\\
&= \langle\mathbf{v_1},\mathbf{w} \rangle + \langle\mathbf{v_2},\mathbf{w} \rangle
\end{align*}

\begin{align*}
\langle \alpha \mathbf{v},\mathbf{w} \rangle &= \alpha v_1 w_1 + \alpha v_2 w_2 + \alpha v_3 w_3 +...+ \alpha v_n w_n \\
&= \alpha(v_1w_1) + \alpha(v_2w_2) + \alpha(v_3w_3) +...+\alpha(v_nw_n) \\
&=\alpha (v_1w_1 + v_2w_2 + v_3w_3 +...+v_nw_n) \\
&= \alpha \langle \mathbf{v}, \mathbf{w} \rangle 
\end{align*}

\begin{align*}
\langle \mathbf{v}, \mathbf{w}_1+\mathbf{w}_2 \rangle &= v_1(w_{1_{1}}+w_{2_{1}}) + v_2(w_{1_{2}}+w_{2_{2}}) + v_3(w_{1_{3}}+w_{2_{3}}) +...+v_n(w_{1_{n}}+w_{2_{n}})  \\
&= v_1w_{1_{1}} + v_1w_{2_{1}} + v_2w_{1_{2}}+ v_2w_{2_{2}} + v_3w_{1_{3}}+ v_3w_{2_{3}} +...+v_nw_{1_{n}}+ v_nw_{2_{n}}\\
&= (v_1w_{1_{1}} + v_2w_{1_{2}} + v_3w_{1_{3}} +...+v_nw_{1_{n}}) +
(v_1w_{2_{1}} + v_2w_{2_{2}} + v_3w_{2_{3}} +...+v_nw_{2_{n}}) \\
&= \langle \mathbf{v}, \mathbf{w}_1 \rangle + \langle \mathbf{v}, \mathbf{w}_2 \rangle
\end{align*}

\begin{align*}
\langle \mathbf{v}, \alpha \mathbf{w} \rangle &= v_1 \alpha w_1 + v_2 \alpha w_2 + v_3 \alpha w_3 +...+ v_n \alpha w_n \\
&= \alpha(v_1w_1) + \alpha(v_2w_2) + \alpha(v_3w_3) +...+\alpha(v_nw_n) \\
&=\alpha (v_1w_1 + v_2w_2 + v_3w_3 +...+v_nw_n) \\
&= \alpha \langle \mathbf{v}, \mathbf{w} \rangle 
\end{align*}

Regarding being positive definite , consider $\langle\mathbf{v}, \mathbf{v}\rangle$
\begin{align*}
\langle\mathbf{v}, \mathbf{v}\rangle &= v_1v_1 + v_2v_2 + v_3v_3 +...+v_nv_n \\
&= v_1^2 + v_2^2 + v_3^2 +...+v_n^2\\
&\geq 0
\end{align*}

Notice also, that $\langle\mathbf{v}, \mathbf{v}\rangle = 0$ if and only if $\mathbf{v} = 0$

\end{proof}

\subsection{Exercise 2.5.11}
Prove the following properties of the norm if $\mathbf{v},\mathbf{w} \in \mathbb{E}^n$

\begin{enumerate}

\item $||\mathbf{v}|| \geq 0$

\begin{proof}
If $\mathbf{v} \in \mathbb{E}^n$ the norm is defined by
\begin{align*}
||\mathbf{v}|| &= \sqrt{\langle \mathbf{v}, \mathbf{v} \rangle}\\
&= \sqrt{(v_1v_1 + v_2v_2 + v_3v_3 +...+v_nv_n)} \\
&= \sqrt{(v_1^2 + v_2^2 + v_3^2 +...+v_n^2)} \\
&= (v_1^2 + v_2^2 + v_3^2 +...+v_n^2)^{\frac{1}{2}} \\
&\geq 0
\end{align*}
\end{proof}

\item $||\mathbf{v}|| = 0$ if and only if $\mathbf{v} = 0$

From the definition of positive definite, $\langle \mathbf{v}, \mathbf{v} \rangle = 0$ if and only if $\mathbf{v} = 0$ 
\begin{proof} "$\Rightarrow$"
Let $\|\mathbf{v}\| = 0$.  Then 
\[ \|\mathbf{v}\| = \sqrt{\langle \mathbf{v}, \mathbf{v} \rangle} = 0 \]
\begin{align*}
\sqrt{\langle \mathbf{v}, \mathbf{v} \rangle} &= 0\\
\sqrt{(v_1v_1 + v_2v_2 + v_3v_3 +...+v_nv_n)} &= 0\\
(v_1^2 + v_2^2 + v_3^2 +...+v_n^2)^{\frac{1}{2}} &= 0 \\
(v_1^2 + v_2^2 + v_3^2 +...+v_n^2)&= 0
\end{align*}
So $\mathbf{v} = 0$
\end{proof}

\begin{proof} "$\Leftarrow$"

Let $\mathbf{v} = 0$

So $\langle \mathbf{v}, \mathbf{v} \rangle = 0$. 

\begin{align*}
\|\mathbf{v}\| &= \sqrt{\langle \mathbf{v}, \mathbf{v} \rangle} \\ 
&= \sqrt{0} \\
&= 0
\end{align*}
\end{proof}

\item $\|\alpha \mathbf{v}\| = |\alpha| \|\mathbf{v}\|$, $\alpha \in \mathbb{r}$ 

\begin{proof}
\begin{align*}
\|\alpha \mathbf{v} \| &= \sqrt{\langle\alpha\mathbf{v},\alpha\mathbf{v}\rangle} \\
&= \sqrt{(\alpha^2v_1^2 +\alpha^2v_2^2 + \alpha^2v_3^2 +...+\alpha^2v_n^2)} \\
&= \sqrt{\alpha^2(v_1^2 + v_2^2 +v_3^2 +...+v_n^2)} \\
&= |\alpha| \sqrt{(v_1^2 + v_2^2 +v_3^2 +...+v_n^2)} \\
&= |\alpha| \sqrt{\langle \mathbf{v}, \mathbf{v} \rangle} \\
&= |\alpha| \|\mathbf{v}\|
\end{align*}
\end{proof}

\item $\|\mathbf{v} + \mathbf{w} \| \leq \|\mathbf{v}\| + \|\mathbf{w}\|$

\begin{proof}

First recall the Pythagorean theorem, $a^2 + b^2 = c^2$ and Cauchy-Schwarz Inequality, let $\mathbf{v}, \mathbf{w} \in \mathbb{E}^n$, then $|\langle \mathbf{v}, \mathbf{w} \rangle| \leq \|\mathbf{v}\|\|\mathbf{w}\|$
\begin{align*}
\|\mathbf{v} + \mathbf{w} \|^2 &= \langle \mathbf{v}+\mathbf{w},\mathbf{v}+\mathbf{w}\rangle \\
&= (v_1+w_1)^2 + (v_2+w_2)^2 +...+(v_n+w_n)^2\\
&= (v_1^2 + v_2^2 +...+v_n^2) + (w_1^2+w_2^2+...+w_n^2) + (v_1w_1+...+v_nw_n) +(w_1v_1+...+w_nv_n) \\ 
&= \langle \mathbf{v},\mathbf{v}\rangle + \langle \mathbf{v}, \mathbf{w} \rangle + \langle \mathbf{w}, \mathbf{v} \rangle + \langle \mathbf{w},\mathbf{w}\rangle \\
&\leq \|\mathbf{v}\|^2 + 2|\langle \mathbf{v}, \mathbf{w} \rangle| + \|\mathbf{w}\|^2 \\
&\leq \|\mathbf{v}\|^2 + 2\|\mathbf{v}\|\|\mathbf{w}\| + \|\mathbf{w}\|^2 \\
&= (\|\mathbf{v}\| + \|\mathbf{w}\|)^2
\end{align*}

\end{proof}

\item $\|\mathbf{v} + \mathbf{w} \|^2 + \|\mathbf{v} -\mathbf{w}\|^2 = 2(\|\mathbf{v}\|^2 + \|\mathbf{w}\|^2)$
\begin{proof}
\begin{align*}
\|\mathbf{v} + \mathbf{w} \|^2 &+ \|\mathbf{v} -\mathbf{w}\|^2 = 
  \langle \mathbf{v} + \mathbf{w},\mathbf{v} + \mathbf{w}\rangle +
  \langle \mathbf{v} - \mathbf{w},\mathbf{v} - \mathbf{w}\rangle \\
&= (v_1 +w_1)^2 + ... + (v_n+w_n)^2 + (v_1-w_1)^2 + ... + (v_n-w_n)^2\\
&= (v_1^2 + 2v_1w_1 + w_1^2 +...+v_n^2 + 2v_nw_n + w_n^2) + (v_1^2 - 2v_1w_1 + w_1^2 +...+v_n^2 - 2v_nw_n + w_n^2) \\
&= (2v_1^2 +...+2v_n^2) + (2w_1^2 +...+ 2w_n^2)\\
&= 2(v_1^2 +...+v_n^2) + 2(w_1^2 +...+w_n^2) \\
&= 2(\langle\mathbf{v},\mathbf{v}\rangle + \langle\mathbf{w},\mathbf{w}\rangle)\\
&= 2(\|\mathbf{v}\|^2 = \|\mathbf{w}\|^2)
\end{align*}
\end{proof}
\end{enumerate}

\subsection{2.5.20}
\begin{enumerate}
\item Show that $\|\vct{v} \times \vct{w} \| = \|\vct{v}\|\|\vct{w}\|\sin \theta$, where $\theta$ is the angle between $\vct{v}$ and $\vct{w}$.

\begin{proof}

First right out the cross product of $\vct{v} \times \vct{w}$

\begin{align}
\vct{v} \times \vct{w} = (v_2w_3 - v_3w_2, v_3w_1 - v_1w_3, v_1w_2 - v_2w_1)
\end{align}

\begin{align*}
\|\vct{v} \times \vct{w} \|^2 &= (v_2w_3 - v_3w_2)^2+(v_3w_1 - v_1w_3)^2+(v_1w_2 - v_2w_1)^2\\
&=v_2^2w_3^2 - 2v_2v_3w_2w_3 + v_3^2w_2^2 \\
&+ v_3^2w_1^2 - 2v_1v_3w_1w_3 + v_1^2w_3^2 \\
&+ v_1^2w_2^2 - 2v_1v_2w_1w_2 + v_2^2w_1^2 \\
\end{align*}

So
\begin{multline}
\|\vct{v} \times \vct{w} \|^2 = v_1^2(w_2^2 + w_3^2) + v_2^2(w_1^2+w_3^2) + v_3^2(w_1^2+w_2^2)\\ 
-2(v_2v_3w_2w_3 + v_1v_3w_1w_3 + v_1v_2w_1w_2)\\
\end{multline}

Recall that $\cos \theta = \frac{\dprod{\vct{v}}{\vct{w}}}{\|\vct{v}\|\vct{w}\|}$ implies $\dprod{\vct{v}}{\vct{w}} = \cos\theta \|\vct{v}\|\|\vct{w}\|$.  Therefore

\begin{align}
\cos^2\theta \|\vct{v}\|^2\|\vct{w}\|^2 = (\dprod{\vct{v}}{\vct{w}})^2
\end{align}

\begin{align}
(\dprod{\vct{v}}{\vct{w}})^2 = (v_1w_1 + v_2w_2 + v_3w_3)^2
\end{align}

\begin{align*}
(v_1w_1 + v_2w_2 + v_3w_3)^2 &= (v_1w_1 + v_2w_2 + v_3w_3)(v_1w_1 + v_2w_2 + v_3w_3)\\
&= v_1^2w_1^2 + v_1v_2w_1w_2 + v_1v_3w_1w_3 \\
&+ v_1^2w_1^2 + v_1v_2w_1w_2 + v_2v_3w_2w_3 \\
&+ v_3^2w_3^2 + v_1v_3w_1w_3 + v_2v_3w_2w_3 \\
\end{align*}

So applying $(3)$ with our result from $(4)$,
\begin{multline}
\cos^2\theta \|\vct{v}\|^2\|\vct{w}\|^2 = v_1^2w_1^2 + v_1^2w_1^2 + v_3^2w_3^2 +\\
	2(v_1v_2w_1w_2 + v_1v_3w_1w_3 + v_2v_3w_2w_3)
\end{multline}

Combining $(4)$ with $(2)$, $\|\vct{v} \times \vct{w} \|^2 +\cos^2\theta \|\vct{v}\|^2\|\vct{w}\|^2$ we get
\begin{align*}
v_1^2w_1^2 &+ v_1^2w_1^2 + v_3^2w_3^2 + 2(v_1v_2w_1w_2 + v_1v_3w_1w_3 + v_2v_3w_2w_3)\\
&+ v_1^2(w_2^2 + w_3^2) + v_2^2(w_1^2+w_3^2) + v_3^2(w_1^2+w_2^2)\\
&-2(v_2v_3w_2w_3 + v_1v_3w_1w_3 + v_1v_2w_1w_2)\\
&= v_1^2w_1^2 + v_1^2w_1^2 + v_3^2w_3^2 \\
&+ v_1^2(w_2^2 + w_3^2) + v_2^2(w_1^2+w_3^2) + v_3^2(w_1^2+w_2^2)\\
&= v_1^2(w_1^2+w_2^2+w_3^2) + v_2^2(w_1^2+w_2^2+w_3^2) + v_3^2(w_1^2+w_2^2+w_3^2)\\
&= (w_1^2+w_2^2+w_3^2)(v_1^2+v_2^2+v_3^2)
\end{align*}

So from this we can conclude
\begin{align}
\|\vct{v} \times \vct{w} \|^2 +\cos^2\theta \|\vct{v}\|^2\|\vct{w}\|^2 = \|\vct{w}\|^2\|\vct{v}\|^2
\end{align}

The result of $(6)$ implies that $\|\vct{v} \times \vct{w} \|^2 = \|\vct{v}\|^2\|\vct{w}\|^2 - \cos^2\theta \|\vct{v}\|^2\|\vct{w}\|^2$.

So our last steps
\begin{align*}
\|\vct{v} \times \vct{w} \|^2 &= \|\vct{v}\|^2\|\vct{w}\|^2 - \cos^2\theta \|\vct{v}\|^2\|\vct{w}\|^2\\
&= \|\vct{v}\|^2\|\vct{w}\|^2 (1-\cos^2\theta)\\
&= \|\vct{v}\|^2\|\vct{w}\|^2(\sin^2\theta)
\end{align*}

We can take the square root to reach our conclusion. 

\begin{align*}
\|\vct{v} \times \vct{w} \| = \|\vct{v}\|\|\vct{w}\|\sin\theta
\end{align*}

\end{proof}
\end{enumerate}

\subsection{Exercise 2.5.26}
Let $\mathbf{v}_1,\mathbf{v}_2,...,\mathbf{v}_{n-1}$ be linearly independent vectors in $\mathbb{E}^n$. Let $\mathbf{p}_0$ be a point of $\mathbb{R}^n$.  Let $\mathbf{H}$ be the hyperplane plane through $\mathbf{p}_0$ spanned by $\mathbf{v}_1,\mathbf{v}_2,...,\mathbf{v}_{n-1}$. If $\mathbf{p}$ is any point in $\mathbb{R}^n$, show that the distance from $\mathbf{p}$ to $\mathbf{H}$, that is, $\inf\{\|\mathbf{p}-\mathbf{q}\| \mid \mathbf{q} \in \mathbf{H} \}$, is given by the length of the vector $\proj_\vct{v}(\vct{p} - \vct{p_0})$ where $\vct{v}$ is the vector obtained in Theorem 2.5.21. Specialize this to obtain formulas for the distance from a point to a line in $\mathbb{R}^2$ and form a point to a plane in $\mathbb{R}^3$.


We are given that $\vct{v}$ is orthogonal to $\vct{v}_1,\vct{v}_2,...,\vct{v}_{n-1}$ and that $\mathbf{H}$ is the hyperplane through $\vct{p}_0$ spanned by $\vct{v}_1,\vct{v}_2,...,\vct{v}_{n-1}$.

Since $\vct{p}$ and $\vct{p}_0$ are points, we can think of $\proj_\vct{v}(\vct{p} - \vct{p_0})$ as a projection of the vector $(\vct{p} - \vct{p}_0)$ onto $\vct{v}$ where $(\vct{p} - \vct{p}_0)$ is the hypotenuse of a right angle formed by $\proj_\vct{v}(\vct{p} - \vct{p_0})$. Which makes the $\proj_\vct{v}(\vct{p} - \vct{p_0})$ the line adjacent to the angle formed by $\proj_\vct{v}(\vct{p} - \vct{p_0})$  which is parallel with $\vct{v}$ which is orthogonal to $\mathbf{H}$.  By the Pythagorean theorem, this must be the shorted distance from $\vct{p}$ to $\mathbf{H}$, because if you took any other distance to the plane, it wouldn't form right angle with the plane and result in a greater distance.



Specializing this for $\mathbb{R}^2$ and $\mathbb{R}^3$ . We consider the $\proj_\vct{v}(\vct{p} - \vct{p_0})$.  Let the vector $\vct{f} = (\vct{p} - \vct{p_0})$.  We only care about the length of the line so we should take it's norm, which is given

\[ \|\proj_\vct{v}(\vct{f})\| = \frac{|\dprod{\vct{f}}{\vct{v}}|}{\|\vct{w}\|} \]

For $\mathbb{R}^2$, define vectors $\vct{v_1}$ and $\vct{v}$.  Let $\vct{v}$  be non zero orthogonal to $\vct{v_1}$.  This means the pair $\{\vct{v_1}, \vct{v}\}$ is a basis for $\mathbb{R}^2$.  Let $\mathbf{H}$ be the plane through $\vct{p_0}$ be a point spanned by $\vct{v_1}$, and contain points $(p_{0_1}, p_{0_2})$.  Let $\vct{p} = (p_1, p_2)$ to be any point in $\mathbb{R}^2$.  by Let $d$ be the distance we are trying to find.

\begin{align*}
d &= \frac{|\dprod{\vct{f}}{\vct{v}}|}{\|\vct{v}\|} \\
&= \frac{|(f_1v_1 + f_2v_2)|}{\|\vct{v}\|} \\
&= \frac{|(p_1 - p_{0_1})v_1 + (p_2-p_{0_2})v_2|}{\|\vct{v}\|}\\
&= \frac{|(p_1 - p_{0_1})v_1 + (p_2-p_{0_2})v_2|}{\sqrt{v_1^2 + v_2^2}}
\end{align*}

For $\mathbb{R}^3$, we have a similar result. define vectors $\vct{v_1},\vct{v_2}$  to be linearly independent.  Let $\vct{v}$ be the determinant of $\vct{v_1},\vct{v_2}$.  Resulting in $\{\vct{v_1},\vct{v_2},\vct{v}\}$ as a basis $\mathbb{R}^3$ Let $\mathbf{H}$ be the plane through the point $\vct{p_0}$  spanned by $\vct{v_1},\vct{v_2}$, and contain points $(p_{0_1}, p_{0_2},p_{0_3})$. Let $\vct{p} = (p_1, p_2, p_3)$ to be any point in $\mathbb{R}^3$.  by Let $d$ be the distance we are trying to find.

\begin{align*}
d &= \frac{|\dprod{\vct{f}}{\vct{v}}|}{\|\vct{v}\|} \\
&= \frac{|(f_1v_1 + f_2v_2 + f_3v_3 )|}{\|\vct{v}\|} \\
&= \frac{|(p_1 - p_{0_1})v_1 + (p_2-p_{0_2})v_2 + (p_3-p_{0_3})v_3|}{\|\vct{v}\|}\\
&= \frac{|(p_1 - p_{0_1})v_1 + (p_2-p_{0_2})v_2+ (p_3-p_{0_3})v_3|}{\sqrt{v_1^2 + v_2^2 + v_3^2}}
\end{align*}

\subsection{Exercise 2.5.30}
Consider the vectors $\vct{v}_1 = (1,1,-1,0)$, $\vct{v}_2=(1,0,0,-1)$ and $\vct{v}_3=(0,1,1,1)$ in $\mathbb{E}^4$.

\begin{enumerate}
\item Use the Gram-Schmidt orthogonalization process on these three vectors to produce a set of three mutually orthogonal vectors that span the same subspace.

Let $\vct{v}_1 = \vct{\tilde{v}}_1$

For $\vct{\tilde{v}}_2 = \vct{v}_2 - \proj_{\vct{\tilde{v}}_1}(\vct{v}_2)$
\begin{align*}
\proj_{\vct{\tilde{v}}_1}(\vct{v}_2) = \frac{\dprod{\vct{v}_2}{\vct{\tilde{v}}_1}}{\|\vct{\tilde{v}}_1\|}\frac{\vct{\tilde{v}}_1}{\|\vct{\tilde{v}}_1\|}
\end{align*}

\begin{align*}
\dprod{\vct{v}_2}{\vct{\tilde{v}}_1} &= (1)(1) + (0)(1) + (0)({-1}) +(-1)(0)\\
&=1 \\
\|\vct{\tilde{v}}_1\| &= \sqrt{1^2+1^2+(-1)^2+0^2} \\
&= \sqrt{3}
\end{align*}

\begin{align*}
\proj_{\vct{\tilde{v}}_1}(\vct{v}_2) &= \frac{1}{\sqrt{3}}\frac{\vct{\tilde{v}}_1}{\sqrt{3}}\\
&= \frac{1}{3}\vct{\tilde{v}_1}\\
&= \frac{1}{3}
\begin{bmatrix}
1&1&-1&0
\end{bmatrix} \\
&=
\begin{bmatrix}
\frac{1}{3} & \frac{1}{3} &-\frac{1}{3} & 0
\end{bmatrix}
\end{align*}

\begin{align*}
\vct{\tilde{v}}_2 &= 
\begin{bmatrix}
1 \\
0 \\
0 \\ 
-1 
\end{bmatrix}
-
\begin{bmatrix}
\frac{1}{3} \\
\frac{1}{3} \\ 
{-\frac{1}{3}} \\
0
\end{bmatrix} \\
&=
\begin{bmatrix}
\frac{2}{3} & -\frac{1}{3} &\frac{1}{3} & -1
\end{bmatrix}
\end{align*}


For $\vct{\tilde{v}}_3 = \vct{v}_3 - \proj_{\vct{\tilde{v}}_1}(\vct{v}_3)- \proj_{\vct{\tilde{v}}_2}(\vct{v}_3)$

\begin{align*}
\proj_{\vct{\tilde{v}}_1}(\vct{v}_3) = \frac{\dprod{\vct{v}_3}{\vct{\tilde{v}}_1}}{\|\vct{\tilde{v}}_1\|}\frac{\vct{\tilde{v}}_1}{\|\vct{\tilde{v}}_1\|}
\end{align*}

\begin{align*}
\dprod{\vct{v}_3}{\vct{\tilde{v}}_1} &= (0)(1) + (1)(1) + (1)(-1) +(1)(0)\\
&=0 \\
\|\vct{\tilde{v}}_1\| &= \sqrt{3}
\end{align*}

\begin{align*}
\proj_{\vct{\tilde{v}}_1}(\vct{v}_3) &= \frac{0}{\sqrt{3}}\frac{\vct{\tilde{v}}_1}{\sqrt{3}}\\
&= 0\vct{\tilde{v}_1}\\
&=0
\begin{bmatrix}
1 & 1 & -1 & 0
\end{bmatrix} \\
&=
\begin{bmatrix}
0 & 0 &0 & 0
\end{bmatrix}
\end{align*}

\begin{align*}
\proj_{\vct{\tilde{v}}_2}(\vct{v}_3) = \frac{\dprod{\vct{v}_3}{\vct{\tilde{v}}_2}}{\|\vct{\tilde{v}}_2\|}\frac{\vct{\tilde{v}}_2}{\|\vct{\tilde{v}}_2\|}
\end{align*}

\begin{align*}
\dprod{\vct{v}_3}{\vct{\tilde{v}}_2} &= (0)(\frac{2}{3}) + (1)({-\frac{1}{3}}) + (1)(\frac{1}{3}) +(1)(-1)\\
&=-1 \\
\|\vct{\tilde{v}}_2\| &= \sqrt{\frac{2}{3}^2+{-\frac{1}{3}}^2+\frac{1}{3}^2+-1^2} \\
&= \sqrt{\frac{5}{3}}
\end{align*}

\begin{align*}
\proj_{\vct{\tilde{v}}_2}(\vct{v}_3) &= -\frac{1}{\sqrt{\frac{5}{3}}}\frac{\vct{\tilde{v}}_2}{\sqrt{\frac{5}{3}}}\\
&= -\frac{3}{5}\vct{\tilde{v}_2}\\
&= -\frac{3}{5}
\begin{bmatrix}
\frac{2}{3} & -\frac{1}{3} &\frac{1}{3} & -1
\end{bmatrix} \\
&=
\begin{bmatrix}
\frac{6}{15} & {-\frac{3}{15}} &\frac{3}{15} & \frac{3}{5}
\end{bmatrix}
\end{align*}

\begin{align*}
\vct{\tilde{v}}_3 &= 
\begin{bmatrix}
0 \\
1 \\
1 \\ 
1 
\end{bmatrix}
-\begin{bmatrix}
0 \\
0\\ 
0\\
0
\end{bmatrix}
-
\begin{bmatrix}
\frac{6}{15} \\
{-\frac{3}{15}} \\ 
\frac{3}{15} \\
\frac{3}{5}
\end{bmatrix} \\
&=
\begin{bmatrix}
{-\frac{6}{15}} & \frac{18}{15} &\frac{12}{15} & {\frac{2}{5}}
\end{bmatrix}
\end{align*}

\item Extend the set of three vectors produced in part $1$ to a mutually orthogonal basis for $\mathbb{4}$

Consider the matrix comprised of $ \vct{\tilde{v}}_1,  \vct{\tilde{v}}_2  \vct{\tilde{v}}_3$ and vectors $\vct{e}_1,\vct{e}_2,\vct{e}_3,\vct{e}_4 \in \mathbb{E}^4$ where $1$ is in the the $jth$ term and $0$ otherwise.

\begin{align}
\begin{bmatrix}
\vct{e}_1 & \vct{e}_2 & \vct{e}_3 & \vct{e}_4 \\
1 & 1 & {-1} & 0 \\
\frac{2}{3} & -\frac{1}{3} &\frac{1}{3} & -1 \\
{-\frac{6}{15}} & \frac{18}{15} &\frac{12}{15} & {\frac{2}{5}}
\end{bmatrix}
\end{align}

The vector that will complete the basis can be created by taking the determinant of the matrix (7).
\end{enumerate} 

Step 1
\begin{multline*}
\vct{e}_1
\begin{bmatrix}
1 & -1 & 0 \\
-\frac{1}{3} &\frac{1}{3} & -1 \\
\frac{18}{15} &\frac{12}{15} & \frac{2}{5}
\end{bmatrix}
-
\vct{e}_2
\begin{bmatrix}
1 & -1 & 0 \\
\frac{2}{3} &\frac{1}{3} & -1 \\
-\frac{6}{15} &\frac{12}{15} & \frac{2}{5}
\end{bmatrix}\\
+\vct{e}_3
\begin{bmatrix}
1 & 1 & 0 \\
\frac{2}{3} & -\frac{1}{3} & -1 \\
-\frac{6}{15} & \frac{18}{15} & \frac{2}{5}
\end{bmatrix}
- \vct{e}_4
\begin{bmatrix}
1 & 1 & -1 \\
\frac{2}{3} & -\frac{1}{3} &\frac{1}{3}\\
-\frac{6}{15} & \frac{18}{15} &\frac{12}{15}
\end{bmatrix} \\
\end{multline*}

Then the next step...

\begin{align*}
&\vct{e}_1(1
\begin{bmatrix}
\frac{1}{3} & -1 \\
\frac{12}{15} & \frac{2}{5}
\end{bmatrix}
- {-1}
\begin{bmatrix}
-\frac{1}{3} & -1 \\
\frac{18}{15} & \frac{2}{5}
\end{bmatrix}
- \\
&\vct{e}_2(1
\begin{bmatrix}
\frac{1}{3} & -1 \\
\frac{12}{15} & \frac{2}{5}
\end{bmatrix} - {-1}
\begin{bmatrix}
\frac{2}{3} & -1 \\
-\frac{6}{15} & \frac{2}{5}
\end{bmatrix}
+\\
&\vct{e}_3(1
\begin{bmatrix}
-\frac{1}{3} & -1 \\
\frac{18}{15} & \frac{2}{5}
\end{bmatrix} 
- 1
\begin{bmatrix}
\frac{2}{3} & -1 \\
-\frac{6}{15} & \frac{2}{5}
\end{bmatrix}
-\\
&\vct{e}_4(1
\begin{bmatrix}
-\frac{1}{3} &\frac{1}{3}\\
\frac{18}{15} &\frac{12}{15}
\end{bmatrix}
-1
\begin{bmatrix}
\frac{2}{3} &\frac{1}{3}\\
-\frac{6}{15} &\frac{12}{15}
\end{bmatrix}
+{-1}
\begin{bmatrix}
\frac{2}{3} & -\frac{1}{3} \\
-\frac{6}{15} & \frac{18}{15}
\end{bmatrix})
\end{align*}

...eventually arriving at...

\begin{align*}
\vct{e}_1(-\frac{2}{5}) - \vct{e}_2(\frac{4}{5}) + \vct{e}_3(\frac{6}{5}) - \vct{e}_4(-2)
\end{align*}

We now compute our new vector to be $(-\frac{2}{5}, -\frac{4}{5}, \frac{6}{5}, 2)$ and define it as $\vct{v}$.  Which gives us the basis $\{\vct{\tilde{v}_1},\vct{\tilde{v}_2},\vct{\tilde{v}_3}, \vct{v}\}$ over $\mathbb{E}^4$.

In order to make this a mutually orthogonal basis for $\mathbb{E}^4$, apply the Gram-Schmidt orthogonalization process once again.

For $\vct{\tilde{v}} = \vct{v} - \proj_{\vct{\tilde{v}}_1}(\vct{v}) - \proj_{\vct{\tilde{v}}_2}(\vct{v})- \proj_{\vct{\tilde{v}}_3}(\vct{v})$

\[
\dprod{\vct{v}}{\vct{\tilde{v}}_1} =  (-\frac{2}{5})(1) + (-\frac{4}{5})(1) + (\frac{6}{5})(-1) + (2)(0) = -\frac{12}{5}
\]

\begin{align*}
\proj_{\vct{\tilde{v}}_1}(\vct{v})  &= -\frac{-\frac{12}{5}}{\sqrt{3}}\frac{\vct{\tilde{v}}_1}{\sqrt{3}}\\
&= -\frac{4}{5}\vct{\tilde{v}}_1\\
&= -\frac{4}{5}
\begin{bmatrix}
1 & 1 & -1 & 0
\end{bmatrix}\\
&= \begin{bmatrix}
-\frac{4}{5} & -\frac{4}{5} & \frac{4}{5} &  0 \\
\end{bmatrix}
\end{align*}

\[
\dprod{\vct{v}}{\vct{\tilde{v}}_2} =  (-\frac{2}{5})(\frac{2}{3}) + (-\frac{4}{5})(-\frac{1}{3}) + (\frac{6}{5})(\frac{1}{3}) + (2)(-1) = -\frac{8}{5}
\]

\[ \|\vct{\tilde{v}}_2\| = \sqrt{\frac{5}{3}} \]
\begin{align*}
\proj_{\vct{\tilde{v}}_2}(\vct{v})  &= -\frac{-\frac{8}{5}}{\sqrt{\frac{5}{3}}}\frac{\vct{\tilde{v}}_2}{\sqrt{\frac{5}{3}}}\\
&= -\frac{24}{25}\vct{\tilde{v}}_2\\
&= -\frac{24}{25}
\begin{bmatrix}
\frac{2}{3} & -\frac{1}{3} & \frac{1}{3} & -1
\end{bmatrix}\\
&= \begin{bmatrix}
-\frac{16}{25} & \frac{8}{25} & -\frac{8}{25} & \frac{24}{25} \\
\end{bmatrix}
\end{align*}

\[
\vct{\tilde{v}} = 
\begin{bmatrix}
-\frac{2}{5}\\
-\frac{4}{5} \\
\frac{6}{5} \\ 
2
\end{bmatrix} -
\begin{bmatrix}
-\frac{4}{5}\\
-\frac{4}{5}\\
\frac{4}{5}\\
0
\end{bmatrix}
-
\begin{bmatrix}
-\frac{16}{25}\\
\frac{8}{25}\\
-\frac{8}{25}\\
\frac{24}{25} \\
\end{bmatrix}
=
\begin{bmatrix}
\frac{11}{13} & \frac{9}{13} &-\frac{7}{13} & \frac{10}{13}
\end{bmatrix}
\]

We now have a mutually orthogonal basis for $\mathbb{E}^4$.
\end{document}
\grid
