\documentclass{tufte-book}

\usepackage{amsmath, amsthm}
\usepackage{graphicx}
\setkeys{Gin}{width=\linewidth,totalheight=\textheight,keepaspectratio}
\graphicspath{{graphics/}}

\title{Real Analysis\\Fifth Week }
\author{Joe Seidel}
\date{\today}

\usepackage{booktabs}
\usepackage{units}
\usepackage{fancyvrb}
\fvset{fontsize=\normalsize}
\usepackage{multicol}
\usepackage{lipsum}
\usepackage{pdfpages}
\usepackage{tikz}
\usepackage{wasysym}
\usepackage{amssymb}

\newcommand{\doccmd}[1]{\texttt{\textbackslash#1}}% command name -- adds backslash automatically
\newcommand{\docopt}[1]{\ensuremath{\langle}\textrm{\textit{#1}}\ensuremath{\rangle}}% optional command argument
\newcommand{\docarg}[1]{\textrm{\textit{#1}}}% (required) command argument
\newenvironment{docspec}{\begin{quote}\noindent}{\end{quote}}% command specification environment
\newcommand{\docenv}[1]{\textsf{#1}}% environment name
\newcommand{\docpkg}[1]{\texttt{#1}}% package name
\newcommand{\doccls}[1]{\texttt{#1}}% document class name
\newcommand{\docclsopt}[1]{\texttt{#1}}% document class option name
\DeclareMathOperator{\proj}{proj}
\newcommand{\vct}{\mathbf}


\newcommand{\dprod}[2]{\langle #1, #2 \rangle}

\newtheoremstyle{mytheoremstyle} % name
	{\topsep}		% Space above
	{\topsep}		% Space below
	{\itshape}		% Body font
	{}			% Indent amount
	{\bfseries}	% Theorem head font
	{\textnormal{:}}	% Punctuation after theorem head
	{.5em}		% Space after theorem head
	{}			%Theorem headspec 
\theoremstyle{mytheoremstyle}
\newtheorem*{thm}{Thm.}

\newtheoremstyle{mylemstyle} % name
	{\topsep}		% Space above
	{\topsep}		% Space below
	{\itshape}		% Body font
	{}			% Indent amount
	{\bfseries}	% Theorem head font
	{\textnormal{:}}	% Punctuation after theorem head
	{.5em}		% Space after theorem head
	{}			%Theorem headspec 
\theoremstyle{mylemstyle}
\newtheorem*{lem}{Lem.}


\newtheoremstyle{mydefstyle} % name
	{\topsep}		% Space above
	{\topsep}		% Space below
	{\normalfont}	% Body font
	{}			% Indent amount
	{\bfseries}	% Theorem head font
	{\textnormal{:}}	% Punctuation after theorem head
	{.5em}		% Space after theorem head
	{}			%Theorem headspec 
\theoremstyle{mydefstyle}
\newtheorem*{mydef}{Def.}
\newtheorem*{ex}{E.g.}

\begin{document}

\maketitle
\pagenumbering{gobble}
\newpage
\pagenumbering{arabic}

\subsection{Exercise 3.3.13}
Let $(X,d)$ be a metric space and let $Y$ be an open set in $X$.  Show that every open set in $(Y,d')$, where $d'$ is the inherited metric, is also open in $X$.

Since $Y$ is an open set, $\forall y_0 \in Y$ and fix $r > 0$ such that $B_r(y_0) \subset Y$. Note that $B_r(y_0) = \{ y \in Y | d'(y,y_0) < r\}$ where $y_0, y \in Y (\subset X)$ and $d'(y_0, y) = d(y_0, y)$.  Since $y_0$ was arbitrary, any open set in $Y$ is also open in $X$.

\subsection{Exercise 3.3.20}
Show that $\mathbb{Q}$ as a subset of $\mathbb{R}$ with the usual metric is neither open or closed in $\mathbb{R}$.  (Of course, if the metric space is simply $\mathbb{Q}$ with the usual metric, then $\mathbb{Q}$ is both open and closed in $\mathbb{Q}$.)

Rational numbers are dense in $\mathbb{R}$, which means between any $p,t \in \mathbb{Q}$, $p \neq t$ there exists an irrational number, $i$.  Without loss of generality say $p < t$, then $p < i < t$.  Therefore, every open ball around $q \in \mathbb{Q}$ contains points not in $\mathbb{Q}$, i.e. the open ball, $\epsilon > 0$, $B_\epsilon(q) \not\subset \mathbb{Q}$.  Therefore $\mathbb{Q}$ is not open.

Similarly, irrational numbers lie between any two rational numbers and none of $\mathbb{Q}^c$ lie entirely in $\mathbb{Q}$.  So, $\mathbb{R}\setminus\mathbb{Q}$ is not open and $\mathbb{Q}$ is not closed.

\subsection{Exercise 3.3.31}
Suppose that $A$ is a subset of a metric space $X$.  Show that $\overline{A} = A \cup \{ \text{accummulation points of }A\}$

\begin{proof}
Let $A'=\{ \text{accummulation points of }A\}$. Consider $A \cup A'$, then any $a \in A \cup A'$ is either in $A$ or $A'$.  Consider $a \in A$, note that $A \subset \overline{A}$, so $a \in \overline{A}$.  If $a \in A'$, since $\overline{A}$ is closed it must contains all the accumulation points of $A$ so $a \in \overline{A}$.  Therefore $A \cup A' \subset \overline{A}$

Now show $\overline{A} \subset A \cup A'$.  Consider any $a \in \overline{A}$.  Then $a \in A$ or $a \in \overline{A}\setminus A$. The first case is trivial. If $a \in \overline{A}\setminus A$, then $a \notin A$ but $a \in \overline{A}$ so $a$ must be an accumulation point of $A$.  Therefore, for any $a \in \overline{A}$, $a \in A$ or $a \in A'$ so conclude $\overline{A} \subset A \cup A'$.
\end{proof}

\subsection{Exercise 3.3.32}
Suppose $A$ is a subset of a metric space $X$.  Prove or disprove $\overline{A} = A\cup \partial A$

\begin{proof}
Want to show $A\cup \partial A \subset \overline{A}$. Consider any $x \in A \cup \partial A$. Then either $x \in A$ or $x \in \partial A$.  If $x \in A$ then $x \in \overline{A}$ since $A \subset \overline{A}$.  If $x \in \partial A$ then for any $r>0$, $B_r(x) \cap A \neq \emptyset$ and $B_r(x) \cap A^c \neq \emptyset$.  This means $x$ is either an isolated point in $A$, so $x \in A$, or it is an accumulation point of $A$.  Again since $A \subset \overline{A}$ and $\overline{A}$ contains all accumulation points, $x \in \overline{A}$.  Since $a$ was arbitrary, $A\cup \partial A \subset \overline{A}$

Next, want to show $\overline{A} \subset A\cup \partial A$.  Consider any $x \in \overline{A}$, then $x \in A$ or $x \notin A$. If $x \in A$ we are done.  If $x \notin A$, then $x \in X \setminus A$.  Suppose $x$ is an exterior point of $A$, then $x \in B_r(x)$, there exists $r >0$ such that $B_r(x) \cap A = \emptyset$. However this contradicts $x \in \overline{A}$ since $\overline{A}$ is the intersection of every closed set containing $A$.  Therefore if $x \notin A$ then $x \in \partial A$.  Then conclude, $\overline{A} \subset A\cup \partial A$.
\end{proof}

\subsection{Exercise 3.3.33}
Suppose A is a subset of a metric space X.  Prove that $\partial A = \overline{A} \cap \overline{A^c}$.

\begin{proof}
Choose any $x \in \overline{A} \cap \overline{A^c}$, then $x \in \overline{A}$ and $x \in \overline{A^c}$. Suppose there exists $r>0$ such that $B_r(x) \cap A = \emptyset$, but then $x \notin \overline{A}$, which is a contradiction since $\overline{A}$ is the intersection of all sets containing $A$.  So $x$ is either in $\partial{A}$ or $A$.  Now we should suppose $x \in A$ and there exists $r > 0$ such that $B_r(x) \subset A$.  However, this contradicts that $x \in \overline{A^c}(= X \setminus \overline{A})$.  Therefore $x \in \partial A$.  This implies, since $x$ was arbitrary, $\overline{A} \cap \overline{A^c} \subset \partial A$.

Now consider any $x \in \partial A$. This implies $\forall r>0$, $B_r(x) \cap A \neq \emptyset$ and $B_r(x) \cap A^c \neq \emptyset$.  Therefore $x \in \overline{A}$.  Since $A^c \subset \overline{A^c}$, $x$ is also in $\overline{A^c}$.  Again, since $x$ was arbitrary $\partial A \subset \overline{A} \cap \overline{A^c}$.

\end{proof}


\subsection{Exercise 3.3.49}
\begin{enumerate}
\item Describe the closed convex hull of the unit ball in $\ell_n^p(\mathbb{R})$ for $1 \leq p \leq \infty$.

Let $B_1(0)$ be the unit ball in $\ell_n^p(\mathbb{R})$ and $\mathcal{I} := \{p,q \in \overline{B_1(0)} \}$.  The closed convex hull of $B_1(0)$ is $\bigcap_{i \in \mathcal{I}} \{(1-t)p_i + t(q_i) | 0 \leq t \leq 1 \text{ with } t \in \mathbb{R} \}$ 

\item Suppose $0 < p < 1$ For $x \in \mathbb{R}^n$, define,
\[ \|x\|_p = \left( \sum_{k=1}^n |x_k|^p \right)^\frac{1}{p} \]

Define $S_p = \{x \in \mathbb{R}^n | \|x\|_p \leq 1 \}$.  Determine whether $S_p$ is convex.  If not, find the closed convex hull of $S_p$.

\begin{proof}

For $S_p$ to be convex, the second derivative needs to be greater than $0$.
\begin{align*}
S_p^{'} &= p|x_k|^{(p-1)}\\
S_p^{''} &= p(p-1)^{''}|x_k|^{(p-2)}
\end{align*}

Since $0< p < 1$, $p-1 < 0$ so $S_p^{''} < 0$ which means it is not convex.  So now it is time to find the closed convex hull of $S_p$.

Consider the points at the corners of $S_p$, i.e. $p,q \in S_p$ where $p=(1,0)$ and $q=(0,1)$ and take the norm of the line segment formed by these two points.

\begin{align*}
\|(1-t)p+tq\|_p &= \|1-t[0,1] + t[0,1]\|_p \text{ with } 0 \leq t \leq 1 \\
&= \|[1-t,0] + [0,t]\|_p\\
&= \|(1-t, t)\|_p \\
&= (|(1-t)|^p + |t|^p)^\frac{1}{p} \\
&\geq ((1-t) + t)^\frac{p}{p} \\
&= 1
\end{align*}

For $0<p<1$, $(|(1-t)|^p + |t|^p)^\frac{1}{p} \geq 1$, so the line segment isn't contained in the $S_p$, confirming the above.  However, for $p=1$ $(|(1-t)|^p + |t|^p)^\frac{1}{p} = 1$ so the closed convex hull of $S_p$ is $S_1 = \{x \in \mathbb{R}^n \text{ | } \|x\|_1 \leq 1 \}$

\end{proof}

\end{enumerate}

\subsection{Not in Book}
Work on the following problems.

\begin{enumerate}
\item $(A^o)^c = \overline{A^c}$
\begin{proof}
Consider any $x \in (A^o)^c$, want so show $(A^o)^c \subset \overline{A^c}$.  Then $x \notin A^o$.  This means $\nexists \epsilon > 0$ such that $B_\epsilon(x) \subset A$.  Therefore $\exists\epsilon > 0$ such that $B_\epsilon(x) \cap A^c \neq 0$, so $x \in \overline{A^c}$.  This implies $(A^o)^c \subset \overline{A^c}$.

Consider any $x \in \overline{A^c}$.  This means $\forall\epsilon >0$, $B_\epsilon(x) \cap A^c \neq \emptyset$.  Which means any open ball around $x$ will intersect $A^c$ so you cannot find an open ball where $B_\epsilon(x) \subset A$.  This implies $x \notin A^o$.  Therefore, $\overline{A^c} \subset (A^o)^c$

\end{proof}

\item An isolated point of A is an accumulation point of $A^c$.

\begin{proof}
Let $x \in A$ be an isolated point of $A$.  This means $\exists\epsilon>0$ such that $B_\epsilon (x) \cap A = \{x\}$.  Since isolated points are boundary points,  it also means, for any $\epsilon >0$ $B_\epsilon(x)\setminus \{x\} \cap A^c \neq \emptyset$.  Therefore, any isolated point of $A$ is an accumulation point of $A^c$.

\end{proof}

\item Construct an example of a set $A$ such that $\overline{A} = \emptyset$.

The Cantor set in space $\mathbb{R}$.  The interior of the Cantor set is empty, i.e. $\mathcal{C}^o = \emptyset$, since it contains no non-empty open intervals.  The closure of $\emptyset$ is also empty, i.e. $\overline{\mathcal{C}^o} = \emptyset$.

\item A set A such that $\overline{A}^{o} = \emptyset$.

Again, the Cantor set fulfils this requirement.  The closure of the Cantor set is the Cantor set.  The interior of the Cantor set is empty.

\end{enumerate}
\end{document}
\grid