\documentclass{tufte-book}

\usepackage{amsmath, amsthm}
\usepackage{graphicx}
\setkeys{Gin}{width=\linewidth,totalheight=\textheight,keepaspectratio}
\graphicspath{{graphics/}}

\title{Real Analysis\\Midterm 2}
\author{Joe Seidel}
\date{\today}

\usepackage{booktabs}
\usepackage{units}
\usepackage{fancyvrb}
\fvset{fontsize=\normalsize}
\usepackage{multicol}
\usepackage{lipsum}
\usepackage{pdfpages}
\usepackage{tikz}
\usepackage{wasysym}

\newcommand{\doccmd}[1]{\texttt{\textbackslash#1}}% command name -- adds backslash automatically
\newcommand{\docopt}[1]{\ensuremath{\langle}\textrm{\textit{#1}}\ensuremath{\rangle}}% optional command argument
\newcommand{\docarg}[1]{\textrm{\textit{#1}}}% (required) command argument
\newenvironment{docspec}{\begin{quote}\noindent}{\end{quote}}% command specification environment
\newcommand{\docenv}[1]{\textsf{#1}}% environment name
\newcommand{\docpkg}[1]{\texttt{#1}}% package name
\newcommand{\doccls}[1]{\texttt{#1}}% document class name
\newcommand{\docclsopt}[1]{\texttt{#1}}% document class option name
\DeclareMathOperator{\proj}{proj}
\newcommand{\vct}{\mathbf}


\newcommand{\dprod}[2]{\langle #1, #2 \rangle}

\newtheoremstyle{mytheoremstyle} % name
	{\topsep}		% Space above
	{\topsep}		% Space below
	{\itshape}		% Body font
	{}			% Indent amount
	{\bfseries}	% Theorem head font
	{\textnormal{:}}	% Punctuation after theorem head
	{.5em}		% Space after theorem head
	{}			%Theorem headspec 
\theoremstyle{mytheoremstyle}
\newtheorem*{thm}{Thm.}

\newtheoremstyle{mylemstyle} % name
	{\topsep}		% Space above
	{\topsep}		% Space below
	{\itshape}		% Body font
	{}			% Indent amount
	{\bfseries}	% Theorem head font
	{\textnormal{:}}	% Punctuation after theorem head
	{.5em}		% Space after theorem head
	{}			%Theorem headspec 
\theoremstyle{mylemstyle}
\newtheorem*{lem}{Lem.}


\newtheoremstyle{mydefstyle} % name
	{\topsep}		% Space above
	{\topsep}		% Space below
	{\normalfont}	% Body font
	{}			% Indent amount
	{\bfseries}	% Theorem head font
	{\textnormal{:}}	% Punctuation after theorem head
	{.5em}		% Space after theorem head
	{}			%Theorem headspec 
\theoremstyle{mydefstyle}
\newtheorem*{mydef}{Def.}
\newtheorem*{ex}{E.g.}

\begin{document}

\maketitle
\pagenumbering{gobble}
\newpage
\pagenumbering{arabic}

\subsection{Question 1}
Prove that the $\ell^p$ norm on $\mathbb{R}^2$ is equivalent to the $\ell^\infty$ norm for all $p \geq 1$.

\begin{proof}
The $\ell^p$ norm of $x$ is $\|x\|_p = (|x_1|^p + |x_2|^p)^{\frac{1}{p}}$ $\forall x = (x_1,x_2) \in \mathbb{R}^2$.  The $\ell^\infty$ norm of $x$ is $\|x\|_\infty = max\{|x_1|, |x_2|\}$.  Suppose, without loss of generality, $\|x\|_\infty = |x_1|$, i.e. $|x_1| \geq |x_2|$.

First 
\[ \|x\|_\infty = |x_1| = (|x_1|^p)^\frac{1}{p} \leq (|x_1|^p + |x_2|^p)^\frac{1}{p} = \|x\|_p \]
which implies $\|x\|_\infty \leq \|x\|_p$.
Next
\[ \|x\|_p = (|x_1|^p + |x_2|^p)^{\frac{1}{p}} \leq (|x_1|^p + |x_1|^p)^\frac{1}{p} = 2^\frac{1}{p}|x_1|^\frac{1}{p} = 2^\frac{1}{p}\|x\|_\infty \]
So
\[ \|x\|_\infty \leq \|x\|_p \leq 2^\frac{1}{p}\|x\|_\infty \]

\end{proof}

\subsection{Question 2}
Suppose $f: X \to X'$ is a bijection (one-to-one and onto) and continuous where $X(\subset \mathbb{R})$ is compact and $X' \subset \mathbb{R}$.  Prove that $f$ is in fact a homeomorphism.

\begin{proof}
It remains to show $f^{-1}$ is continuous.  We need to show for any open set $U \subset X$, $(f^{-1})^{-1}(U) = f(U)$ is open in $X'$.  Equivently for any closed set $V \subset X$, $(f^{-1})^{-1}(V) = f(V)$  is closed in $X'$.

Since $X$ is compact, we have any closed subset $V(\subset X)$ is compact (midterm 1).  Next $f$ is continuous implies that $f$ maps compact sets to compact sets. So $f(V)$ is compact.  Since $X'$ is bounded in $\mathbb{R}$, $f(V)$ as a compact set in $\mathbb{R}$ is bounded and closed.  This shows that for any closed set $V \subset X$, its image $f(V)$ is closed.  This shows $f^{-1}$ is continuous, so $f$ is a homeomorphism.
\end{proof}

\subsection{Question 3}
Show the sequence
\[ \{ \cos^nx \,|\, x \in [-\frac{\pi}{2}, \frac{\pi}{2}] \} \]
does not converge uniformly.

\begin{proof}
A simple proof is by the Dini Theorem.  If $\cos^nx$ for $x \in [-\frac{\pi}{2}, \frac{\pi}{2}]$ converges uniformly, by Dini theorem, the limiting function should be continuous.  However, the pointwise limit is
\[ \cos^nx \to f(x) = \begin{cases}
      1, & \text{if}\ x = 0\\
      0, & \text{if}\ x \neq 0\\
\end{cases} 
\]
discontinous.
\end{proof}

\begin{proof}
Or we can show $\exists \epsilon$ such that for all $n$, there exists $x_n$ such that $|\cos^nx_n - f(x_n)| > \epsilon$.  We choose $\epsilon = \frac{1}{2}$  Since $f(x) = 0$ for $x \neq 0$ it is enough to find $x_n$ satisfying $\cos x_n > (\frac{1}{2})^\frac{1}{n}$.  Since $0 < (\frac{1}{2})^\frac{1}{n} < 1$, such $x_n$ always exists.

\end{proof}

\subsection{Question 4}
Find the closure, interior and boundary of the following sets.
\begin{enumerate}
\item The interval $(0,1)$ as a subset of $\mathbb{C}$.\\
Closure $[0,1] \subset \mathbb{C}$.  Interior $\emptyset$.  Boundary $[0,1] \subset \mathbb{C}$

\item The set of rational numbers $\mathbb{Q}$ as a subset of $\mathbb{R}$\\
Closure $\mathbb{R}$. Interior $\emptyset$. Boundary $\mathbb{R}$.

\item The Cantor set as a subset of $\mathbb{R}$\\
Closure Cantor set.  Interor $\emptyset$.  Boundary Cantor set.
\end{enumerate}

\subsection{Question 5}
Consider two vectors $v_1=(1,1,0)$ and $v_2=(3,0,4)$ in $\mathbb{R}^3$ endowed with the standard inner product.  The two vectors span a plane
\[ P = \{sv_1 + tv_2 \,|\, s,t \in \mathbb{R} \}. \]
Use Gram-Schmidt to produce an orthonormal basis for the plane P.

$\tilde{v}_1 = v_1$ and $\tilde{v}_2 = v_2 - \proj_{v_1}(v_1)$
Then normalize $v_1$ and $\tilde{v}_2$.
\subsection{Question 6}
Consider a metric space $(X,d)$.  Suppose both two sets $S_1, S_2 \subset X$ are open and dense in $X$.  Prove that $S_1 \cap S_2$ is open and dense in $X$.

\begin{proof}
The intersection of finite open sets is open, so $S_1 \cap S_2$ is open.  To show $S_1 \cap S_2$ is dense, we consider any nonempty open set $U \subset X$.  Since $S_1$ is dense, we have $S_1 \cap U \neq \emptyset$.  Pick $x \in S_1 \cap U$, since $S_1 \cap U$ is open, we have that $\exists \epsilon$ such that $B_\epsilon(x) \subset S_1 \cap U$.  Since $S_2$ is dense, we get $B_\epsilon(x) \cap S_2 \neq \emptyset$.  This implies $B_\epsilon(x) \cap S_2 \subset S_1 \cap S_2 \cap U$.  Hence $S_1 \cap S_2$ is dense.
\end{proof}

\subsection{Question 7}
We introduce the metric $d(x,y) = \frac{|x-y|}{1+ |x-y|}$ on $\mathbb{R}$.  Show that $\mathbb{R}$ is complete under this metric.  You do not need to prove that $d$ is a metric.

\begin{proof}

It is enough to show any Cauchy sequence has a limit in $\mathbb{R}$.  Suppose $\{x_n\}$ is a Cauchy sequence in the new metric, i.e. $\forall \epsilon$ $\exists N$ such that when $m,n > N$ we have
\[ d(x_m, x_n) = \frac{|x_m - x_n|}{1+|x_m-x_n|} < \epsilon \].

Then we have $|x_m - x_n| < \epsilon + |x_m-x_n|\epsilon$ for $\epsilon < \frac{1}{2}$, we have $|x_m-x_n| < \frac{\epsilon}{1-\epsilon} < 2\epsilon$.   This implies $\forall \epsilon < \frac{1}{2}$ $\exists N$ such that when $m, n > N$ we have $|x_m - x_n| < 2\epsilon$.  Hence $\{x_n\}$ is a Cauchy sequence in $\mathbb{R}$ in the usual metric. $\mathbb{R}$ is complete in this metric, so $\exists x \in \mathbb{R}$ such that $\lim_{n \to \infty}x_n = x$, in the usual metric.

Furthermore, $d(x_n, x) = \frac{|x_n -x|}{1+|x_n-x|} < |x_n-x|$.  Therefore $\{x_n\}$ converges to $x$ also in the new metric.

\end{proof}

\end{document}
\grid