\documentclass{tufte-book}

\usepackage{amsmath, amsthm}
\usepackage{graphicx}
\setkeys{Gin}{width=\linewidth,totalheight=\textheight,keepaspectratio}
\graphicspath{{graphics/}}

\title{Real Analysis}
\author{Joe Seidel}
\date{\today}

\usepackage{booktabs}
\usepackage{units}
\usepackage{fancyvrb}
\fvset{fontsize=\normalsize}
\usepackage{multicol}
\usepackage{lipsum}
\usepackage{pdfpages}
\usepackage{tikz}

\newcommand{\doccmd}[1]{\texttt{\textbackslash#1}}% command name -- adds backslash automatically
\newcommand{\docopt}[1]{\ensuremath{\langle}\textrm{\textit{#1}}\ensuremath{\rangle}}% optional command argument
\newcommand{\docarg}[1]{\textrm{\textit{#1}}}% (required) command argument
\newenvironment{docspec}{\begin{quote}\noindent}{\end{quote}}% command specification environment
\newcommand{\docenv}[1]{\textsf{#1}}% environment name
\newcommand{\docpkg}[1]{\texttt{#1}}% package name
\newcommand{\doccls}[1]{\texttt{#1}}% document class name
\newcommand{\docclsopt}[1]{\texttt{#1}}% document class option name


\newtheoremstyle{mytheoremstyle} % name
	{\topsep}		% Space above
	{\topsep}		% Space below
	{\itshape}		% Body font
	{}			% Indent amount
	{\bfseries}	% Theorem head font
	{\textnormal{:}}	% Punctuation after theorem head
	{.5em}		% Space after theorem head
	{}			%Theorem headspec 
\theoremstyle{mytheoremstyle}
\newtheorem*{thm}{Thm.}

\newtheoremstyle{mylemstyle} % name
	{\topsep}		% Space above
	{\topsep}		% Space below
	{\itshape}		% Body font
	{}			% Indent amount
	{\bfseries}	% Theorem head font
	{\textnormal{:}}	% Punctuation after theorem head
	{.5em}		% Space after theorem head
	{}			%Theorem headspec 
\theoremstyle{mylemstyle}
\newtheorem*{lem}{Lem.}


\newtheoremstyle{mydefstyle} % name
	{\topsep}		% Space above
	{\topsep}		% Space below
	{\normalfont}	% Body font
	{}			% Indent amount
	{\bfseries}	% Theorem head font
	{\textnormal{:}}	% Punctuation after theorem head
	{.5em}		% Space after theorem head
	{}			%Theorem headspec 
\theoremstyle{mydefstyle}
\newtheorem*{mydef}{Def.}
\newtheorem*{ex}{E.g.}

\begin{document}

\maketitle
\pagenumbering{gobble}
\newpage
\pagenumbering{arabic}

\section{Section 1.5 Construction of the Real Numbers}

\subsection{Exercise 1.5.1} 

Show that for any $a, b \in \mathbb{Q}$, we have $||a|-|b|| \leq |a-b|$.

\begin{proof} Since $a,b \in \mathbb{Q}$, \marginnote{Absolute values on $\mathbb{Q}$ satisfy the Triangle Inequality}
\[|a+b| \leq |a|+|b|\]
So
\[|a|=|a-b+b| \leq |a-b| + |b|\]
\[|b|=|a+b-a| \leq |b-a| + |a|\]
These can be rewritten as 
\[|a|-|b| \leq |a-b|\]
\[|b|-|a| \leq |b-a|\]
Since $|a-b|=|b-a|$ and if $t \geq x$ and $t \geq -x$ then $t \geq |x|$, therefore 
\[||a|-|b|| \leq |a-b| \]\end{proof}

\subsection{Exercise 1.5.5}
If a sequence $(a_k)_{k \in \mathbb{N}}$ converges in $\mathbb{Q}$ show that 
$(a_k)_{k \in \mathbb{N}}$ is a Cauchy sequence in $\mathbb{Q}$.

\begin{proof}By definition if $(a_k)_{k \in \mathbb{N}}$ converges in $\mathbb{Q}$ given any rational number $r > 0$ there exists an integer $N$ such that if $n \geq N$ then $|a_n-a|<r$. 

Suppose $(a_k)_{k \in \mathbb{N}}$ converges to $a, a \in \mathbb{Q}$. Let $r>0$, since $(a_k)_{k \in \mathbb{N}}$ converges to $a$,  $\exists N$ such that $\forall n \geq N$, $|a_n-a| < \frac{r}{2}$.

Then $\forall n,m > N$
\[|a_n-a_m| = |a_n-a+a-a_m| \leq |a_n-a| + |a-a_m|\]

Since $n > N$ and $m > N$
\[|a_n-a| < \frac{r}{2}\]

and
\[|a-a_m| = |a_m-a| < \frac{r}{2}\]

therefore
\[|a_n-a_m|< \frac{r}{2} + \frac{r}{2} = r\]
\end{proof}

\end{document}
\grid
