\documentclass{tufte-book}

\usepackage{amsmath, amsthm}
\usepackage{graphicx}
\setkeys{Gin}{width=\linewidth,totalheight=\textheight,keepaspectratio}
\graphicspath{{graphics/}}

\title{Real Analysis}
\author{Joe Seidel}
\date{\today}

\usepackage{booktabs}
\usepackage{units}
\usepackage{fancyvrb}
\fvset{fontsize=\normalsize}
\usepackage{multicol}
\usepackage{lipsum}
\usepackage{pdfpages}
\usepackage{tikz}
\usepackage{wasysym}

\newcommand{\doccmd}[1]{\texttt{\textbackslash#1}}% command name -- adds backslash automatically
\newcommand{\docopt}[1]{\ensuremath{\langle}\textrm{\textit{#1}}\ensuremath{\rangle}}% optional command argument
\newcommand{\docarg}[1]{\textrm{\textit{#1}}}% (required) command argument
\newenvironment{docspec}{\begin{quote}\noindent}{\end{quote}}% command specification environment
\newcommand{\docenv}[1]{\textsf{#1}}% environment name
\newcommand{\docpkg}[1]{\texttt{#1}}% package name
\newcommand{\doccls}[1]{\texttt{#1}}% document class name
\newcommand{\docclsopt}[1]{\texttt{#1}}% document class option name


\newtheoremstyle{mytheoremstyle} % name
	{\topsep}		% Space above
	{\topsep}		% Space below
	{\itshape}		% Body font
	{}			% Indent amount
	{\bfseries}	% Theorem head font
	{\textnormal{:}}	% Punctuation after theorem head
	{.5em}		% Space after theorem head
	{}			%Theorem headspec 
\theoremstyle{mytheoremstyle}
\newtheorem*{thm}{Thm.}

\newtheoremstyle{mylemstyle} % name
	{\topsep}		% Space above
	{\topsep}		% Space below
	{\itshape}		% Body font
	{}			% Indent amount
	{\bfseries}	% Theorem head font
	{\textnormal{:}}	% Punctuation after theorem head
	{.5em}		% Space after theorem head
	{}			%Theorem headspec 
\theoremstyle{mylemstyle}
\newtheorem*{lem}{Lem.}


\newtheoremstyle{mydefstyle} % name
	{\topsep}		% Space above
	{\topsep}		% Space below
	{\normalfont}	% Body font
	{}			% Indent amount
	{\bfseries}	% Theorem head font
	{\textnormal{:}}	% Punctuation after theorem head
	{.5em}		% Space after theorem head
	{}			%Theorem headspec 
\theoremstyle{mydefstyle}
\newtheorem*{mydef}{Def.}
\newtheorem*{ex}{E.g.}

\begin{document}

\maketitle
\pagenumbering{gobble}
\newpage
\pagenumbering{arabic}

\section{Section 1.5 Construction of the Real Numbers}

\subsection{Exercise 1.5.1} 

Show that for any $a, b \in \mathbb{Q}$, we have $||a|-|b|| \leq |a-b|$.

\begin{proof} Since $a,b \in \mathbb{Q}$, \marginnote{Absolute values on $\mathbb{Q}$ satisfy the Triangle Inequality}
\[|a+b| \leq |a|+|b|\]
So
\[|a|=|a-b+b| \leq |a-b| + |b|\]
\[|b|=|a+b-a| \leq |b-a| + |a|\]
These can be rewritten as 
\[|a|-|b| \leq |a-b|\]
\[|b|-|a| \leq |b-a|\]
Since $|a-b|=|b-a|$ and if $t \geq x$ and $t \geq -x$ then $t \geq |x|$, therefore 
\[||a|-|b|| \leq |a-b| \]\end{proof}

\subsection{Exercise 1.5.5}
If a sequence $(a_k)_{k \in \mathbb{N}}$ converges in $\mathbb{Q}$ show that 
$(a_k)_{k \in \mathbb{N}}$ is a Cauchy sequence in $\mathbb{Q}$.

\begin{proof}By definition if $(a_k)_{k \in \mathbb{N}}$ converges in $\mathbb{Q}$ given any rational number $r > 0$ there exists an integer $N$ such that if $n \geq N$ then $|a_n-a|<r$. 

Suppose $(a_k)_{k \in \mathbb{N}}$ converges to $a, a \in \mathbb{Q}$. Let $r>0$, since $(a_k)_{k \in \mathbb{N}}$ converges to $a$,  $\exists N$ such that $\forall n \geq N$, $|a_n-a| < \frac{r}{2}$.

Then $\forall n,m > N$
\[|a_n-a_m| = |a_n-a+a-a_m| \leq |a_n-a| + |a-a_m|\]

Let $n,m > N$ 
\[|a_n-a| < \frac{r}{2}\]

and
\[|a-a_m| = |a_m-a| < \frac{r}{2}\]

therefore
\[|a_n-a_m|< \frac{r}{2} + \frac{r}{2} = r\]
\end{proof}

\subsection{Exercise 1.5.6}
Show that the limit of a convergent sequence is unique.

\begin{proof}Suppose $(a_k)_{k \in \mathbb{Q}}$ converges in $\mathbb{Q}$ to $L$ and $M$. Choose $L$ and $M , L \neq M$ and let $r=\frac{|L-M|}{2}$. Then $\exists N_1 \in \mathbb{Z}$ such that if $n \geq N_1$ 
then
\[|a_n-L|<r\]
and $\exists N_2 \in \mathbb{Z}$ such that if $n \geq N_2$ then
\[|a_n-M|<r\]
Let $N=max(N_1,N_2)$. If $n \geq N$ then 
\[|L-M| = |l-a_n + a_n - M| \leq |L-a_n| + |a_n-M| < 2(\frac{|L-M}{2}|) = |L-M|\]

Reducing the above, we have $|L-M| < |L-M|$  a contradiction, $\Rightarrow \Leftarrow$.  Therefore, $L=M$.

\end{proof}

\subsection{Exercise 1.5.9}
Show that the sum of two Cauchy sequences in $\mathbb{Q}$ is a Cauchy sequence in $\mathbb{Q}$.

\begin{proof}Let $(a_k)_{k \in \mathbb{N}}$ and $(b_k)_{k \in \mathbb{N}}$ be Cauchy sequences $\mathbb{Q}$. Let $r>0$, $\exists N_1$ such that if $n,m \geq N_1$ then 
\[|a_n - a_m| < \frac{r}{2}\]
and $\exists N_2$ such that if $n,m \geq N_2$ then 
\[|b_n - b_m| < \frac{r}{2}\]
Let $N=max(N_1,N_2)$. If $n,m \geq N$ then
\[|a_n - a_m| + |b_n - b_m| < \frac{r}{2} + \frac{r}{2} = r\]
\[|a_n - a_m + b_n - b_m| \leq |a_n - a_m| + |b_n - b_m|\]
Therefore
\[|(a_n + b_n) - (a_m + b_m)| < r\]

\end{proof}

\subsection{Exercise 1.5.13}
Show that if a Cauchy sequence $(a_k)_{k  \in \mathbb{N}}$ does not converge to $0$, all the terms of the sequence eventually have the same sign.


\begin{lem}{1.5.12} Suppose $(a_k)_{k  \in \mathbb{N}} \in \mathcal{C} \setminus \mathcal{I}$  \marginnote{Where $\mathcal{C}$ denotes the set of all Cauchy sequences of rational numbers and $\mathcal{I}$ denotes the set of all Cauchy sequences that converge to $0$.  }, then there exists a positive rational number $r$ and an integer $N$ such that $|a_n| \geq r$ for all $n \geq N$.
\end{lem}
\begin{proof} Suppose $(a_k)_{k \in \mathbb{N}}$ is a Cauchy sequence that does not converge to $0$. Therefore given any $r>0$ there exists an integer $N$ such that if $n,m \geq N$, then $|a_n - a_m| < r$.   From Lemma 1.5.2, we can choose $r >0$ and $N$ such that $|a_n| \geq r$ for all $n \geq N$. 

Let $r > 0$ and $n,m \geq N$. Therefore
\[|a_n - a_m| < r \leq |a_n|\] 
Suppose  $a_n > 0$ and $a_m < 0$\marginnote{In other words, they don't have the same sign.}
\[|a_n-(-a_m)| = |a_n + a_m| < |a_n|  \Rightarrow \Leftarrow \]  
Likewise, suppose $a_n < 0$ and $a_m > 0$
\[|a_n-a_m| = |a_m-a_n|\]
\[|a_m - (-a_n)| = |a_m+a_n| < |a_n|  \Rightarrow \Leftarrow \]
Therefore, all terms must eventually be the same sign.

\end{proof}


\subsection{Exercise 1.5.15}
Show that $\sim$ defines an equivalence relation on $\mathcal{C}$
We need to show reflexivity, symmetry, and transitivity exist on Cauchy sequences that are equivalent.
\[(a_k) \sim (a_k)\]
For all $a_n \in (a_k), |a_n-a_n| = 0$ This we can say that $(a_k-a_k)_{k \in \mathbb{N}}$ is in $\mathcal{I}$
\[(a_k) \sim (b_k)\]
Suppose that $(a_k)$ and $(b_k)$ are equivalent and $r > 0$. For all $N \in \mathbb{N}$ There exists $|(a_n)-(b_n)|<r$ and $|(b_n)-(a_n)|<r$ for $n \geq N$ 
\[(a_k) \sim (b_k) \text{,} (b_k) \sim (c_k) \Rightarrow (a_k) \sim (c_k)\]
Let $r>0$, $\exists N_1 \in \mathbb{N}$ such that $|a_n - b_n| < \frac{r}{2}$ for all $n \geq N$ and $\exists N_2 \in \mathbb{N}$ such that $|b_n - c_n| < \frac{2}{2}$ for all $n \geq N_2$.  This implies 
\[ |a_n - c_c| = |a_n - b_n + b_n-c_n| \leq |a_n - b_n| + |b_n - c_n| < \frac{r}{2} + \frac{r}{2} = r \]
for all $n = max(N_1,N_2)$

\subsection{Exercise 1.5.17}
Show that \textbf{R} is a commutative ring with 1, with $\mathcal{I}$ as the additive identity and $[a_k]$ such that $a_k=1$ for all $k$ as the multiplicative identity.

We know that if $(a_k),(b_k)$ are Cauchy sequences, $(a_n)_{n \in \mathbb{N}} +(a_n)_{n \in \mathbb{N}}=(a_n + b_n)_{n \in \mathbb{N}}$ and $(a_n)_{n \in \mathbb{N}}(b_n)_{n \in \mathbb{N}}=(a_nb_n)_{n \in \mathbb{N}}$ are well-defined.
Let $[a_k]$ be an equivalence class, it easily follows that $[a_k]+[a_k] = [a_k+a_k]$ and $[a_k][a_k] = [a_k a_k]$

As examples, consider $(a_k)_{k \in \mathbb{N}}$ and $(a'_k)_{k \in \mathbb{N}}$ denoted as $\{a_k\}$ and $\{a'_k\}$, respectively. Likewise, $(b_k)_{k \in \mathbb{N}}$ and $(b'_k)_{k \in \mathbb{N}}$ denoted as $\{b_k\}$ and $\{b'_k\}$  

For addition, let $\{a_k\} \sim \{a'_k\}$ , $\{b_k\} \sim \{b'_k\}$ and $r>0$.  Then, $\exists N_1$ $\exists N_2$ in $\mathbb{N}$ such that
\[|a_n-a'_n| < \frac{r}{2} \text{ for } n \geq N_1 \]
and
\[|b_n-b'_n| < \frac{r}{2} \text{ for } n \geq N_2 \]
This implies
\[|(a_n+b_n) - (a'_n + b'_n)| = |a_n - a'_n + b_n - b'_n| \geq |a_n-a'n| - |b_n - b'n| < \frac{r}{2} + \frac{r}{2} \] 
for $n \geq max(N_1,N_2)$.

If $[i_k]$ is in $\mathcal{I}$ then $\{i_k\}+\{i_k\}=0$ for all $k \in \mathbb{N}$ Then it easily follows $[a_k] + [i_k] = [a_k]$

\marginnote{\begin{lem}{1.5.8. Let $(a_k)_{k \in \mathbb{N}}$ be a Cauchy sequence of rational numbers. Then $(a_k)_{k \in \mathbb{N}}$ is a bounded sequence } \end{lem}}
For multiplication, recall that $\{a_k\}$, $\{a'_k\}$, $\{b_k\}$ and $\{b'_k\}$ are bounded. $\exists M >0$ such that  $\{a_n\}$, $\{a'_n\}$, $\{b_n\}$ ,$\{b'_n\} \leq M$ for all $n \in \mathbb{N}$ $\exists N \in \mathbb{N}$ such that
\[|a_n - a'_n| < \frac{r}{2M} \text{ for } n \geq N_1 \]
and
\[|b_n - b'_n| < \frac{r}{2M} \text{ for } n \geq N_2 \]

\begin{align*}
2|a_nb_n - a'_nb'_n| & = |(a_n-a'_n)(b_n+b'_n) + (a_n+a'_n)(b_n-b'_n)|
\\& \leq |(a_n-a'_n)(b_n+b'_n)| + |(a_n+a'n)(b_n-b'_n)|
\\& = |a_n-a'n||b_n+b'_n| + |a_n+a'n||b_n-b'_n|
\\& \leq |a_n-a'n|(|b_n|+|b'_n|) + (|a_n|+|a'n|)|b_n-b'_n|
\\& < \frac{r}{2M}(2M) + \frac{r}{2M}(2M)
\\& = 2r
\end{align*}
Therefore $|a_nb_n - a'nb'_n| < r$ for all $n = max(N_1,N_2)$.  If $[i_k]$  with $i_k=1$ for all $k$ is the multiplicative identity it follows that $[a_k][i_k] = [a_k]$


\subsection{Exercise 1.5.20}
Show that order relation, defined below is well-defined and makes \textbf{R} and ordered field. 

\begin{mydef}Let $a = [a_k]$ and $b=[b_k]$ be distinct elements of \textbf{R}. We define $a<b$ if $a_k < b_k$ eventually and $b>a$ if $b_k > a_k$ eventually.
\end{mydef}

Let $r > 0$


Consider $a$ and $b$ as denoted above. From what is already know about about $a$ and $b$ we say

\[|a_n - b_n| < r \text{ for some } n \geq N \in \mathbb{N}\] 

Let them be distinct elements of \textbf{R}. Since $a$ and $b$ are not equivalent, $a-b$ is not in $\mathcal{I}$. Therefore either
\[|a_n - b_n| < r < |a_n| \text{ or } |a_n - b_n| < r < |b_n| \]

Consider
\[|a_n-b_n| < |a_n|\]
in which case
\[0 < |a_n| - r < |a_n-b_n| < |a_n| .\]
Likewise,
\[|a_n-b_n| < |b_n|\]
in which case
\[0 < |b_n| - r < |a_n-b_n| < |b_n| .\]
We've shown in Exercise 1.5.13 that $a_n$ and $b_n$ must have the same sign. If $a_n=b_n$ in either case, then we'll arrive at a contradiction.  Therefore, either $a_k > b_k$ or $a_k<b_k$ eventually.
\begin{enumerate}
\item $(O1)$ \textbf{Trichotomy:}  Since $[a] - [b]$ is not in $\mathcal{I}$, by definition either $a < b$ or $b > a$.

\item $(O2)$ \textbf{Transitivity:} For sake of argument, let $a < b$ and choose an additional arbitrary element of \textbf{R} $[c_k] = c$. Let $b < c$. Then $a < c$

\item $(03)$ \textbf{Addition:} Let $a < b$ and choose $c$ to be $\mathcal{I}$ it easily follows that $a + c < a + b$

\item $(04)$ \textbf{Multiplication:} $a < b$ and let $c$ be the multiplicative\\ identity $c_k=1$ for all $k \in \mathbb{N}$
\end{enumerate}


\subsection{Exercise 1.6.11}
Find a bounded sequence of real numbers that is not convergent.

Define $(a_k)_{k \in \mathcal{N}} = (-1)^k$, this sequence is bounded $[-1,1]$.  It is clear that $\{1,-1,1,-1,...\}$ does not converge.

\subsection{Exercise 1.6.16}
Prove Lemma 1.5.15
\begin{lem}Lemma 1.5.15 Every bounded sequence in $\mathbb{R}$ has a convergent subsequence\end{lem} 

Let $(a_k)_{k \in \mathbb{N}}$ be a bounded sequence in $\mathbb{R}$.

\begin{lem}Lemma 1.6.13: every bounded sequence in $\mathbb{R}$ has a monotonic subsequence.
\end{lem}

\begin{lem}Lemma 1.6.14: Every bounded monotonic sequence in $\mathbb{R}$ converges to an element in $\mathbb{R}$. 
\end{lem}

If $(a_k)_{k \in \mathbb{N}}$ does not have a monotonically increasing subsequence, $\exists n_1 \in \mathbb{N}$ such that $a_{n_{1}} > a_k$ for $k > n_1$. It follows that since $(a_k)_{k>n}$ is not monotonically increasing, there exists $a_{n_{2}} > a_k$ for $k > n_2$ and $a_{n_{1}} > a_{n_{2}}$ 
This process can be repeated over the set $(a_k)_{k \in \mathbb{N}}$ to create a strictly monotonic decreasing set $(a_{n_{1}}, a_{n_{2}}, ..., a_{n_{k}})$.

Alternatively, if $(a_k)_{k \in \mathbb{N}}$ does not have a strictly monotonic decreasing subsequence.  We say $a_{n_{1}} < a_k$  for $k \geq n_1$ Repeating steps above to form a set $(a_{n_{1}}, a_{n_{2}}, ..., a_{n_{k}})$. Which is monotonic increasing.

By definition 1.6.9 a subsequence $(a_{k_{n}}$ is bounded because $(a_{k})$, Is bounded.

Finally, $(a_{k_{n}}$ is monotonic and bounded and therefore converges to an element in ${R}$.

If increasing $\exists N$ such that $a \geq a_k \geq a_N > a - \epsilon$  If decreasing $\exists N$ such that $a \geq a_N \geq a_k > a - \epsilon$.


\end{document}
\grid
