\documentclass{tufte-book}

\usepackage{amsmath, amsthm}
\usepackage{graphicx}
\setkeys{Gin}{width=\linewidth,totalheight=\textheight,keepaspectratio}
\graphicspath{{graphics/}}

\title{Real Analysis}
\author{Joe Seidel}
\date{\today}

\usepackage{booktabs}
\usepackage{units}
\usepackage{fancyvrb}
\fvset{fontsize=\normalsize}
\usepackage{multicol}
\usepackage{lipsum}
\usepackage{pdfpages}
\usepackage{tikz}
\usepackage{ skull }

\newcommand{\doccmd}[1]{\texttt{\textbackslash#1}}% command name -- adds backslash automatically
\newcommand{\docopt}[1]{\ensuremath{\langle}\textrm{\textit{#1}}\ensuremath{\rangle}}% optional command argument
\newcommand{\docarg}[1]{\textrm{\textit{#1}}}% (required) command argument
\newenvironment{docspec}{\begin{quote}\noindent}{\end{quote}}% command specification environment
\newcommand{\docenv}[1]{\textsf{#1}}% environment name
\newcommand{\docpkg}[1]{\texttt{#1}}% package name
\newcommand{\doccls}[1]{\texttt{#1}}% document class name
\newcommand{\docclsopt}[1]{\texttt{#1}}% document class option name


\newtheoremstyle{mytheoremstyle} % name
	{\topsep}		% Space above
	{\topsep}		% Space below
	{\itshape}		% Body font
	{}			% Indent amount
	{\bfseries}	% Theorem head font
	{\textnormal{:}}	% Punctuation after theorem head
	{.5em}		% Space after theorem head
	{}			%Theorem headspec 
\theoremstyle{mytheoremstyle}
\newtheorem*{thm}{Thm.}

\newtheoremstyle{mylemstyle} % name
	{\topsep}		% Space above
	{\topsep}		% Space below
	{\itshape}		% Body font
	{}			% Indent amount
	{\bfseries}	% Theorem head font
	{\textnormal{:}}	% Punctuation after theorem head
	{.5em}		% Space after theorem head
	{}			%Theorem headspec 
\theoremstyle{mylemstyle}
\newtheorem*{lem}{Lem.}


\newtheoremstyle{mydefstyle} % name
	{\topsep}		% Space above
	{\topsep}		% Space below
	{\normalfont}	% Body font
	{}			% Indent amount
	{\bfseries}	% Theorem head font
	{\textnormal{:}}	% Punctuation after theorem head
	{.5em}		% Space after theorem head
	{}			%Theorem headspec 
\theoremstyle{mydefstyle}
\newtheorem*{mydef}{Def.}
\newtheorem*{ex}{E.g.}

\begin{document}

\maketitle
\pagenumbering{gobble}
\newpage
\pagenumbering{arabic}

\section{Section 1.5 Construction of the Real Numbers}

\subsection{Exercise 1.5.1} 

Show that for any $a, b \in \mathbb{Q}$, we have $||a|-|b|| \leq |a-b|$.

\begin{proof} Since $a,b \in \mathbb{Q}$, \marginnote{Absolute values on $\mathbb{Q}$ satisfy the Triangle Inequality}
\[|a+b| \leq |a|+|b|\]
So
\[|a|=|a-b+b| \leq |a-b| + |b|\]
\[|b|=|a+b-a| \leq |b-a| + |a|\]
These can be rewritten as 
\[|a|-|b| \leq |a-b|\]
\[|b|-|a| \leq |b-a|\]
Since $|a-b|=|b-a|$ and if $t \geq x$ and $t \geq -x$ then $t \geq |x|$, therefore 
\[||a|-|b|| \leq |a-b| \]\end{proof}

\subsection{Exercise 1.5.5}
If a sequence $(a_k)_{k \in \mathbb{N}}$ converges in $\mathbb{Q}$ show that 
$(a_k)_{k \in \mathbb{N}}$ is a Cauchy sequence in $\mathbb{Q}$.

\begin{proof}By definition if $(a_k)_{k \in \mathbb{N}}$ converges in $\mathbb{Q}$ given any rational number $r > 0$ there exists an integer $N$ such that if $n \geq N$ then $|a_n-a|<r$. 

Suppose $(a_k)_{k \in \mathbb{N}}$ converges to $a, a \in \mathbb{Q}$. Let $r>0$, since $(a_k)_{k \in \mathbb{N}}$ converges to $a$,  $\exists N$ such that $\forall n \geq N$, $|a_n-a| < \frac{r}{2}$.

Then $\forall n,m > N$
\[|a_n-a_m| = |a_n-a+a-a_m| \leq |a_n-a| + |a-a_m|\]

Let $n,m > N$ 
\[|a_n-a| < \frac{r}{2}\]

and
\[|a-a_m| = |a_m-a| < \frac{r}{2}\]

therefore
\[|a_n-a_m|< \frac{r}{2} + \frac{r}{2} = r\]
\end{proof}

\subsection{Exercise 1.5.6}
Show that the limit of a convergent sequence is unique.

\begin{proof}Suppose $(a_k)_{k \in \mathbb{Q}}$ converges in $\mathbb{Q}$ to $L$ and $M$. Choose $L$ and $M , L \neq M$ and let $r=\frac{|L-M|}{2}$. Then $\exists N_1 \in \mathbb{Z}$ such that if $n \geq N_1$ 
then
\[|a_n-L|<r\]
and $\exists N_2 \in \mathbb{Z}$ such that if $n \geq N_2$ then
\[|a_n-M|<r\]
Let $N=max(N_1,N_2)$. If $n \geq N$ then 
\[|L-M| = |l-a_n + a_n - M| \leq |L-a_n| + |a_n-M| < 2(\frac{|L-M}{2}|) = |L-M|\]

Reducing the above, we have $|L-M| < |L-M|$ $ \skull$, a contradiction.  Therefore, $L=M$.

\end{proof}

\subsection{Exercise 1.5.9}
Show that the sum of two Cauchy sequences in $\mathbb{Q}$ is a Cauchy sequence in $\mathbb{Q}$.

\begin{proof}Let $(a_k)_{k \in \mathbb{N}}$ and $(b_k)_{k \in \mathbb{N}}$ be Cauchy sequences $\mathbb{Q}$. Let $r>0$, $\exists N_1$ such that if $n,m \geq N_1$ then 
\[|a_n - a_m| < \frac{r}{2}\]
and $\exists N_2$ such that if $n,m \geq N_2$ then 
\[|b_n - b_m| < \frac{r}{2}\]
Let $N=max(N_1,N_2)$. If $n,m \geq N$ then
\[|a_n - a_m| + |b_n - b_m| < \frac{r}{2} + \frac{r}{2} = r\]
\[|a_n - a_m + b_n - b_m| \leq |a_n - a_m| + |b_n - b_m|\]
Therefore
\[|(a_n + b_n) - (a_m + b_m)| < r\]

\end{proof}

\subsection{Exercise 1.5.13}
Show that if a Cauchy sequence $(a_k)_{k  \in \mathbb{N}}$ does not converge to $0$, all the terms of the sequence eventually have the same sign.


\begin{lem}{1.5.12} Suppose $(a_k)_{k  \in \mathbb{N}} \in \mathcal{C} \setminus \mathcal{I}$  \marginnote{Where $\mathcal{C}$ denotes the set of all Cauchy sequences of rational numbers and $\mathcal{I}$ denotes the set of all Cauchy sequences that converge to $0$.  }, then there exists a positive rational number $r$ and an integer $N$ such that $|a_n| \geq r$ for all $n \geq N$.
\end{lem}
\begin{proof} Suppose $(a_k)_{k \in \mathbb{N}}$ is a Cauchy sequence that does not converge to $0$. Therefore given any $r>0$ there exists an integer $N$ such that if $n,m \geq N$, then $|a_n - a_m| < r$.   From Lemma 1.5.2, we can choose $r >0$ and $N$ such that $|a_n| \geq r$ for all $n \geq N$. 

Let $r > 0$ and $n,m \geq N$. Therefore
\[|a_n - a_m| < r \leq |a_n|\] 
Suppose  $a_n > 0$ and $a_m < 0$\marginnote{In other words, they don't have the same sign.}
\[|a_n-(-a_m)| = |a_n + a_m| < |a_n| \skull  \] 
Likewise, suppose $a_n < 0$ and $a_m > 0$
\[|a_n-a_m| = |a_m-a_n|\]
\[|a_m - (-a_n)| = |a_m+a_n| < |a_n| \skull \]
Therefore, all terms must eventually be the same sign.

\end{proof}

\end{document}
\grid
