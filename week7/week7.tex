\documentclass{tufte-book}

\usepackage{amsmath, amsthm}
\usepackage{graphicx}
\setkeys{Gin}{width=\linewidth,totalheight=\textheight,keepaspectratio}
\graphicspath{{graphics/}}

\title{Real Analysis\\Sixth Week }
\author{Joe Seidel}
\date{\today}

\usepackage{booktabs}
\usepackage{units}
\usepackage{fancyvrb}
\fvset{fontsize=\normalsize}
\usepackage{multicol}
\usepackage{lipsum}
\usepackage{pdfpages}
\usepackage{tikz}
\usepackage{wasysym}
\usepackage{amssymb}

\newcommand{\doccmd}[1]{\texttt{\textbackslash#1}}% command name -- adds backslash automatically
\newcommand{\docopt}[1]{\ensuremath{\langle}\textrm{\textit{#1}}\ensuremath{\rangle}}% optional command argument
\newcommand{\docarg}[1]{\textrm{\textit{#1}}}% (required) command argument
\newenvironment{docspec}{\begin{quote}\noindent}{\end{quote}}% command specification environment
\newcommand{\docenv}[1]{\textsf{#1}}% environment name
\newcommand{\docpkg}[1]{\texttt{#1}}% package name
\newcommand{\doccls}[1]{\texttt{#1}}% document class name
\newcommand{\docclsopt}[1]{\texttt{#1}}% document class option name
\DeclareMathOperator{\proj}{proj}
\newcommand{\vct}{\mathbf}


\newcommand{\dprod}[2]{\langle #1, #2 \rangle}

\newtheoremstyle{mytheoremstyle} % name
	{\topsep}		% Space above
	{\topsep}		% Space below
	{\itshape}		% Body font
	{}			% Indent amount
	{\bfseries}	% Theorem head font
	{\textnormal{:}}	% Punctuation after theorem head
	{.5em}		% Space after theorem head
	{}			%Theorem headspec 
\theoremstyle{mytheoremstyle}
\newtheorem*{thm}{Thm.}

\newtheoremstyle{mylemstyle} % name
	{\topsep}		% Space above
	{\topsep}		% Space below
	{\itshape}		% Body font
	{}			% Indent amount
	{\bfseries}	% Theorem head font
	{\textnormal{:}}	% Punctuation after theorem head
	{.5em}		% Space after theorem head
	{}			%Theorem headspec 
\theoremstyle{mylemstyle}
\newtheorem*{lem}{Lem.}


\newtheoremstyle{mydefstyle} % name
	{\topsep}		% Space above
	{\topsep}		% Space below
	{\normalfont}	% Body font
	{}			% Indent amount
	{\bfseries}	% Theorem head font
	{\textnormal{:}}	% Punctuation after theorem head
	{.5em}		% Space after theorem head
	{}			%Theorem headspec 
\theoremstyle{mydefstyle}
\newtheorem*{mydef}{Def.}
\newtheorem*{ex}{E.g.}

\begin{document}

\maketitle
\pagenumbering{gobble}
\newpage
\pagenumbering{arabic}

\subsection{Exercise 3.5.9}
Suppose that $(X,d)$ and $(X',d')$ are metric spaces and that $f: X \to X'$ is continuous.  For each of the following statements, determine whether or not is true.  If the assertion is true, prove it.  If it is not true, give a counter example.

\begin{enumerate}

\item If $A$ is an open subset of $X$, then $f(A)$ is an open subset of $X'$;\\
Not neccessarily true.  Consider the constant function $f: \mathbb{R} \to \mathbb{R}$, $f(x) = c$.  Let $A$ be an open subset of $\mathbb{R}$, then $f(A)$ is a closed subset of $\mathbb{R}$.

\item If $A$ is a closed subset of $X$, then $f(A)$ is a closed subset of $X'$;\\
Not neccessarily true.  Consider the function $f: \mathbb{R} \to \mathbb{R}$, $f(x) = \frac{x}{x+1}$.  If $A = [0, \infty)$ then $f(A) = [0,1)$ which is not closed.

\item If $B$ is a closed subset of $X'$, then $f^{-1}(B)$ is a closed subset of $X$;\\
True.  First note that $f^{-1}(S^c) = (f^{-1}(S))^c$.  Since $B \subset X'$ is closed, $B^c \subset X'$ is open.  From Theorem 3.5.5. a function $f: X \to X'$ is continuous iff for any open set $V \in X'$, the set $f^{-1}(V)$ is open in $X$.  Thefore, if $B^c$ is open then $f^{-1}(B^c)$ is open so $f^{-1}(B^c) = (f^{-1}(B))^c$ then $((f^{-1}(B))^c)^c = (f^{-1}(B))$ is closed.

\item If $A$ is a bounded subset of $X$, then $f(A)$ is a bounded subset of $X'$;


\end{enumerate}

\subsection{Exercise 3.5.13}

\subsection{Exercise 3.5.15}

\subsection{Exercise 3.5.23(sans isometry part)}

\subsection{Exercise 3.5.30}

\subsection{Exercise 3.5.33}


\end{document}
\grid