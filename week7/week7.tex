\documentclass{tufte-book}

\usepackage{amsmath, amsthm}
\usepackage{graphicx}
\setkeys{Gin}{width=\linewidth,totalheight=\textheight,keepaspectratio}
\graphicspath{{graphics/}}

\title{Real Analysis\\Seventh Week }
\author{Joe Seidel}
\date{\today}

\usepackage{booktabs}
\usepackage{units}
\usepackage{fancyvrb}
\fvset{fontsize=\normalsize}
\usepackage{multicol}
\usepackage{lipsum}
\usepackage{pdfpages}
\usepackage{tikz}
\usepackage{wasysym}
\usepackage{amssymb}

\newcommand{\doccmd}[1]{\texttt{\textbackslash#1}}% command name -- adds backslash automatically
\newcommand{\docopt}[1]{\ensuremath{\langle}\textrm{\textit{#1}}\ensuremath{\rangle}}% optional command argument
\newcommand{\docarg}[1]{\textrm{\textit{#1}}}% (required) command argument
\newenvironment{docspec}{\begin{quote}\noindent}{\end{quote}}% command specification environment
\newcommand{\docenv}[1]{\textsf{#1}}% environment name
\newcommand{\docpkg}[1]{\texttt{#1}}% package name
\newcommand{\doccls}[1]{\texttt{#1}}% document class name
\newcommand{\docclsopt}[1]{\texttt{#1}}% document class option name
\DeclareMathOperator{\proj}{proj}
\newcommand{\vct}{\mathbf}


\newcommand{\dprod}[2]{\langle #1, #2 \rangle}

\newtheoremstyle{mytheoremstyle} % name
	{\topsep}		% Space above
	{\topsep}		% Space below
	{\itshape}		% Body font
	{}			% Indent amount
	{\bfseries}	% Theorem head font
	{\textnormal{:}}	% Punctuation after theorem head
	{.5em}		% Space after theorem head
	{}			%Theorem headspec 
\theoremstyle{mytheoremstyle}
\newtheorem*{thm}{Thm.}

\newtheoremstyle{mylemstyle} % name
	{\topsep}		% Space above
	{\topsep}		% Space below
	{\itshape}		% Body font
	{}			% Indent amount
	{\bfseries}	% Theorem head font
	{\textnormal{:}}	% Punctuation after theorem head
	{.5em}		% Space after theorem head
	{}			%Theorem headspec 
\theoremstyle{mylemstyle}
\newtheorem*{lem}{Lem.}


\newtheoremstyle{mydefstyle} % name
	{\topsep}		% Space above
	{\topsep}		% Space below
	{\normalfont}	% Body font
	{}			% Indent amount
	{\bfseries}	% Theorem head font
	{\textnormal{:}}	% Punctuation after theorem head
	{.5em}		% Space after theorem head
	{}			%Theorem headspec 
\theoremstyle{mydefstyle}
\newtheorem*{mydef}{Def.}
\newtheorem*{ex}{E.g.}

\begin{document}

\maketitle
\pagenumbering{gobble}
\newpage
\pagenumbering{arabic}

\subsection{Exercise 3.5.9}
Suppose that $(X,d)$ and $(X',d')$ are metric spaces and that $f: X \to X'$ is continuous.  For each of the following statements, determine whether or not is true.  If the assertion is true, prove it.  If it is not true, give a counter example.

\begin{enumerate}

\item If $A$ is an open subset of $X$, then $f(A)$ is an open subset of $X'$;\\
Not neccessarily true.  Consider the constant function $f: \mathbb{R} \to \mathbb{R}$, $f(x) = c$.  Let $A$ be an open subset of $\mathbb{R}$, then $f(A)$ is a closed subset of $\mathbb{R}$.

\item If $A$ is a closed subset of $X$, then $f(A)$ is a closed subset of $X'$;\\
Not neccessarily true.  Consider the function $f: \mathbb{R}_+ \to \mathbb{R}$, $f(x) = \frac{x}{x+1}$.  If $A = [0, \infty)$ then $f(A) = [0,1)$ which is not closed.

\item If $B$ is a closed subset of $X'$, then $f^{-1}(B)$ is a closed subset of $X$;\\
True.  First note that $f^{-1}(S^c) = (f^{-1}(S))^c$.  Since $B \subset X'$ is closed, $B^c \subset X'$ is open.  From Theorem 3.5.5. a function $f: X \to X'$ is continuous iff for any open set $V \in X'$, the set $f^{-1}(V)$ is open in $X$.  Thefore, if $B^c$ is open then $f^{-1}(B^c)$ is open so $f^{-1}(B^c) = (f^{-1}(B))^c$ then $((f^{-1}(B))^c)^c = (f^{-1}(B))$ is closed.

\item If $A$ is a bounded subset of $X$, then $f(A)$ is a bounded subset of $X'$;\\
False,  Consider $f: \mathbb{R} \to \mathbb{R}$ where $f(x) = \frac{1}{x}$.  Take the bonded subset, $A = (0,1)$, however $\lim_{x \to 0_+} = \infty$.  Therefore $\forall M>0$ $\exists \delta$ such that $|x| < \delta$ implies $|f(x)| > M$.  In particular, $\forall n \in \mathbb{N}$, $\exists x_n$ such that $f|(x_n)| > n$ hence $f(A)=(1, \infty)$ is unbounded.

\item If $B$ is a bounded subset of $X'$, then $f^{-1}(B)$ is a bounded subset of $X$.\\
False, define $f: \mathbb{R}_+ \to \mathbb{R}$ $f(x) = \frac{x}{x+1}$. Suppose $f^{-1}(B) = [0, \infty)$ then $f(f^{-1}(B)) = B = (0,1)$ which is bounded.

\item If $A \subset X$ and $x_0$ is an isolated point of $A$, then $x_0$ is an isolated point of $A$;\\
False.  Consider the function $f: \mathbb{R} \to \mathbb{R}$, $f(x) = x^2$ and a subset of $\mathbb{R} \supset \{-1\} \cup [0,2]= A$.  Take the isolated point $-1$ in $A$ and note that $f(-1) = 1$ which is not isolated since $f(A) = [0,4]$.

\item If $A \subset X$, $x_0 \in X$ and $f(x_0)$ is an isolated point of $f(A)$, then $x_0$ is an isolated point of $A$;\\
False.  Consider, again, the constant function $f(x) = c$.  Chose any $x \subset A (\subset X)$.  Suppose $A$ is open.  $f(x) = c$ which is an isolated point since $\exists\epsilon$ such that $B_\epsilon(f(x)) \setminus f(x) \cap f(A) = \emptyset$.  Since $A$ is open, $x$ is not an isolated point of $A$.

\item If $A \subset X$ and $x_0$ is an accumulation point of $A$, then $x_0$ is an accumulation point of $f(A)$.\\
False, consider the same example as above.  Let $x \in A$ and $x$ is an accumulations, however $f(x)$ is an isolated point of $f(A)$.

\item If $A \subset X$, $x_0 \in X$, and $f(x_0)$ is an accumulation point of $f(A)$, then $x_0$ is an accumulation point of $A$.\\
False.  Consider the example used in item $6$.  Since $f(x_0) = 1$ is accumulation but $x_0$ can be $1$ or $-1$.  So $x_0$ is not necessarily an accumulation point.

\end{enumerate}

\subsection{Exercise 3.5.13}
Let $X = [0,1)$ with the induced metric from $\mathbb{R}$, and let $X' = \mathbb{T} = \{z \in \mathbb{C}\text{ | } |z| = 1 \}$ with the induced metric for $\mathbb{C}$. The function $f : X \to X'$, $f(x) = e^{2\pi i x}$ is a continuous bijection whose inverse is not continuous.

An alternative form of the function can be written $f(x) = \cos(2 \pi x) + i \sin(2\pi x)$.   Over the domain $X=[0,1)$ this function is continuous, one to one and onto.  Making it a continuous bijection.  However, it's inverse function is not continuous since $X' = \mathbb{T} = \{z \in \mathbb{C}\text{ | } |z| = 1 \}$ is closed while $X$ is not closed.  Which is a corollary to Theorem 3.5.5, a function is continuous iff for any open set $V \subset X'$, the set $f^-1(V)$ is an open set in $X$.

\subsection{Exercise 3.5.15}
Let $X=R$ with the discrete metric, and let $X' = \mathbb{R}$ with the usual metric.  Show that function $I: X \to X'$, $I(x) = x$ is a continuous bijection but is not a homeomorphism.

\begin{proof}
To show $I(x)$ is a continuous bijection we need to find that it is continuous, one to one, and onto.   The identity function is continuous since give any $\epsilon >0$ we can find a $\delta >0$ such that $d'(x,y) <  \delta$ implies $d'(I(x), I(y)) < \epsilon$.    Let $\epsilon > 0$, we can choose $d = \epsilon + 1$ since $d(x,y) \leq 1$ for any $x, y \in X$, $d(x, y) < \delta$.  Hence $I$ is continuous.  Since $I^{-1}(I(x)) = I(I^{-1}(x))=x$ the function is one to one and onto.  Therefore it is a continuous bijection.

Now we examine homeomorphism.  A function is a homeomorphism if $f$ is continuous, $f$ is bijective,  and $f^{-1}$ is continuous.   Since everything "disconnected" in discrete space, suspect that the function is not continuous. To see that the inverse function of this function is not continuous, choose any $\delta >0$ and let $\epsilon = \frac{1}{2}$.   $0<d(x, y) < \delta$ implies $d'(x, y) = 1 > \epsilon$.  Hence, we have shown that there exists an $\epsilon$ such that for any $\delta >0$, $d'(x,y) > \epsilon$ and hence $I^{-1}$ is not continuous.

\end{proof}
\subsection{Exercise 3.5.23(sans isometry part)}
In  this exercise, we consider isometries from from $\mathbb{R}$ to itself in the usual metric.
\begin{enumerate}
\item Is $f(x) = x^3$ a bijection? A homeomorphism? An Isometry?

Since, $f^{-1}(f(x)) f^{-1}(x^3) = (x^3)^\frac{1}{3} = x$ and $f(f^{-1}(x))  = f(x^\frac{1}{3}) =(x^\frac{1}{3})^3 = x$, $f(x)$ is one to one.  Since $\forall y \in \mathbb{R}'$ There exists $x \in \mathbb{R}$ such that $f(x) = y$ and  so the function is onto, hence it is a bijection.   Since the product of continuous functions is continuous, and $x * x* x= x^3$, $f(x)$ is continuous.  Also not that $f^{-1}$
\begin{align*}
|\sqrt[3]{x}-\sqrt[3]{a}| &= |\sqrt[3]{x}-\sqrt[3]{a}| \times \frac {| x^{\frac 2 3} + \sqrt[3]{a}\sqrt[3]{x} + a^{\frac 2 3} |}{|x^{\frac 2 3} + \sqrt[3]{a}\sqrt[3]{x} + a^{\frac 2 3} |}\\
&= \frac {|x - a|}{|x^{\frac 2 3} + \sqrt[3]{a}\sqrt[3]{x} + a^{\frac 2 3} |} \le \frac {| x - a |}{| {ax} |^{\frac 1 3}}
\end{align*}
Assume $|x-a| < |a|$ then $|x| < 2|a|$ which implies $|ax| < 2|a|^2$ so  $\frac {1}{2|a|^2} < \frac {1} {|ax|}$.

Therefore $|x^{\frac{1}{3}} - a^{\frac{1}{3}}| < \frac{|x-a|}{|{a}|^{\frac{2}{3}}}$ whenever $|x - a| < |a|$ and $a \neq 0$.  Pick $\delta = min\{|a|, \epsilon|a|^\frac{2}{3}\}$, or in the case that $a = 0$ pick $\delta = \epsilon^3$.  These values for delta imply continuity.

Hence, this function is a homeomorphism. 

This function is not an isometry since $d(x,y) \neq d'(f(x), f(y))$, $\forall x,y \in \mathbb{R}$.  This can be verified by picking $x = 2$ and $y=3$.

\item If $f(x) = x + \sin x$ a bijection? A homeomorphism? An isometry?\\
First check to see if $f(x)$ is one to one an onto.  Consider $sin(x) + x = sin(y) + y$, this can be reduced to $x=y$ so for any $x,y \in \mathbb{R}$ $f(x) = f(y)$ implies $x = y$. So it is one to one.  The function is also continuous since $sin$ is a continuous function and $x$ is a continuous function.  Like wise, the inverse function $f^{-1}(x) = cos(x) - x$ is continuous.  Hence, it is a bijection and homeomorphism.

To see that this function is not isometric.  We can consider points $y=2$ and $x= 4$ $d(x,y) = 2$, but $d'(f(x) + f(y)) = |(sin(2) + 2) - (sin(4) + 4)| \neq 2$.

\end{enumerate}

\subsection{Exercise 3.5.30}
Define a sequence of functions $f_n:(0,1) \to \mathbb{R}$ by
\[f_n(x) = 
\begin{cases}
      \frac{1}{q^n}, & \text{if}\ x = \frac{p}{q} \text{ (reduced to lowest terms) }\\
      0, & \text{otherwise}\\
\end{cases}
\]
for $n \in \mathbb{N}$.  Find the pointwise limit $f$ of the sequence $(f_n)_{n \in \mathbb{N}}$ and show that $(f_n)_{n \in \mathbb{N}}$ converges to $f$ uniformly.

\begin{proof}
For any $x \in (0,1)$, either $x = \frac{p}{q}$(reduced to lowest terms), in otherwords $x \in \mathbb{Q}$, or $x$ is irrational.   

When $x \in \mathbb{Q}$, $f_n(x) = \frac{1}{q^n}$.  Since $\lim_{n \to \infty}\frac{1}{a^n}=0$ when $a > 0$, the sequence $f_n(x)$ converges to $0$.   Likewise, when $x \not\in \mathbb{Q}$, $f_n(x)$ converges to $0$.  Therefore, the pointwise limit of $f$ can be defined $f(x) = 0$.

A sequence converges uniformly if given $\epsilon >0$ $\exists N_\epsilon \in \mathbb{N}$ such that $sup_{x \in X}|f_n(x) - f(x)| < \epsilon$ whenever $n > N_\epsilon$.  

Since $f(x) = 0$ for any $x \in (0,1)$ given any $\epsilon$ we can choose $N_\epsilon$ large enough such that either $|f_n(x)| < \epsilon$ when $x$ is rational and when $x$ is irrational $0 < \epsilon$.  Also note that although $(0, 1)$ contains infitiley many points, eventually, the value for $f_n(x) = \frac{1}{q^n}$ or $f_n(x) = 0$ becomes constant.  In other words, putting aside the exponent $n$, there are only finitely many values either, $\frac{1}{q}$ or ${0}$ in the range of the function.

\end{proof}
\subsection{Exercise 3.5.33}
Let $X = (0, \infty)$ and determine whether the following functions are uniformly continuous on $X$.

\begin{enumerate}

\item $f(x) = \frac{1}{x}$

Suppose $f$ is uniformly continous on $(0, \infty)$.  Then given any $\epsilon > 0$, $\exists \delta$ such that if $x>0$ and $y>0$ then $|x - y| < \delta$ implies $|f(x) - f(y)| < \epsilon$.

Set $x_n = \frac{1}{n}$ and $y_n=\frac{1}{2n}$.  Then $|x_n - y_n| = \frac{1}{2n} <  \delta$ for $n$ large enough.  Then, $|f(x_n) - f(y_n)| = \frac{|x_n-y_n|}{x_ny_n} = n > \epsilon$ for large $n$.

Therefore, the fucntion is not uniformly continuous.

\item $f(x) = \sqrt{x}$

Let $\delta = \epsilon^2$.  If $|x - y| < \delta = \epsilon^2$ we have $|\sqrt{x} - \sqrt{y}|^2 \leq |\sqrt{x} - \sqrt{y}||\sqrt{x} + \sqrt{y}| = |x-y| < \epsilon^2$.  Hence $|\sqrt{x} - \sqrt{y}| < \epsilon$.  Thus $f$ is continuous on the interval $(0, \infty)$.

\item $f(x) = \ln(x)$\\
Suppose $f$ is uniformly continous on $(0, \infty)$.  Then given any $\epsilon > 0$, $\exists \delta$ such that if $x>0$ and $y>0$ then $|x - y| < \delta$ implies $|f(x) - f(y)| < \epsilon$.

Set $x_n = \frac{1}{n}$ and $y_n = \frac{1}{2n}$.   $|x_n - y_n| = \frac{1}{2n} <  \delta$ for $n$ large enough.  Then, $|f(x_n) - f(y_n)| = \ln(\frac{2}{n^2}) > \epsilon$ for large $n$.

Therefore, the fucntion is not uniformly continuous.

\item $f(x) = x \ln(x)$
Suppose $f$ is uniformly continous on $(0, \infty)$.  Then given any $\epsilon > 0$, $\exists \delta$ such that if $x>0$ and $y>0$ then $|x - y| < \delta$ implies $|f(x) - f(y)| < \epsilon$.

Set $x_n = n$ and $y_n = n+1$.   $|x_n - y_n| = 1$ for any $n$, but $|f(x_n) - f(y_n)| = n \ln(\frac{1}{n} + 1) + \ln(n+1) > \epsilon$ for large $n$.

So $f$ is not uniformly continous over $(0, \infty)$.
\end{enumerate}

\end{document}
\grid